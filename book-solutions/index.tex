\documentclass[a4paper,twoside,openright]{book}
\usepackage[greek,british]{babel}

% ams maths
\usepackage{amsmath}
\usepackage{amssymb}
\usepackage{amsthm}

% bibliography
\usepackage[numbers]{natbib}
%\renewcommand{\bibname}{Further Reading}
\addto{\captionsbritish}{\renewcommand{\bibname}{Further Reading}}
\usepackage[nottoc]{tocbibind}

\usepackage{tikz}
\usepackage{pgfplots}
\usepackage{hyperref}
\usetikzlibrary{arrows,shapes,positioning,automata}
\tikzstyle{triangle}=[inner sep=0cm,fill opacity=0.6, minimum height = 0.9cm, regular polygon,  regular polygon sides=3]
\tikzstyle{square}=[fill opacity=0.6, minimum height = 0.4cm, regular polygon,  regular polygon sides=4]
\tikzstyle{mycircle}=[fill opacity=0.6, minimum size = 0.4cm, circle]

% tables
\usepackage{longtable}
\usepackage{booktabs}

%\usepackage{fontawesome}
\usepackage{awesomebox}
\setlength{\aweboxleftmargin}{9mm}
\setlength{\aweboxsignraise}{-5mm}
\newcommand{\personbox}[2]{\awesomebox[cyan]{2pt}{\faUser}{cyan}{#1}}
\newcommand{\examplebox}[2]{\awesomebox[Green]{2pt}{\faLightbulb}{Green}{#1}}
\newcommand{\applicationbox}[2]{\awesomebox[cyan]{2pt}{\faDesktop}{cyan}{#1}}

%% fonts
%https://tex.stackexchange.com/questions/50586/how-to-use-palatino-like-math-fonts-in-xelatex
\usepackage{fontspec}
%\usepackage{palatino}
\usepackage[sc]{mathpazo}
\usepackage[mathscr]{eucal}
%\setmainfont[
%    Path = fonts/,
%    Extension = .ttc,
%%  BoldFont = *-Bold,
%%  ItalicFont = *-Italic,
%%  BoldItalicFont = *-BoldItalic,
%    Ligatures=TeX
%%  %SmallCapsFont={TeX Gyre Termes},
%%  %SmallCapsFeatures={Letters=SmallCaps},
%]{Palatino}
\setmainfont
     [ BoldFont       = texgyrepagella-bold.otf ,
       ItalicFont     = texgyrepagella-italic.otf ,
       BoldItalicFont = texgyrepagella-bolditalic.otf ]
     {texgyrepagella-regular.otf}
%\usepackage{unicode-math}
%\setmathfont[
%    Path = fonts/,
%    Extension = .ttf
%]{CambriaMath}
%% instead of package contour, we use these becaus xelatex: see https://tex.stackexchange.com/questions/421970/contour-text-in-xelatex

% problem environment, for exercises
\newcounter{problemcounter}
\newcounter{partcounter}[problemcounter]

\newcommand{\Item}[1]{\par\hangafter=0
                         \hangindent=15pt
                         \noindent\llap{#1}\ignorespaces}
\newcommand{\IItem}[1]{\par\hangafter=0
                         \hangindent=40pt
                         \noindent\llap{#1}\ignorespaces}

\newcommand{\problem}{\smallskip\stepcounter{problemcounter}\Item{\bfseries\arabic{problemcounter}.\ }}
\newfontfamily\daggerfont[Ligatures=TeX]{Garamond-Math.otf}
\newcommand{\solutionmark}{{\daggerfont\textdagger}}
%\newcommand{\problemsol}{\smallskip\stepcounter{problemcounter}\Item{(\solutionmark)\bfseries\arabic{problemcounter}.\ }}
\newcommand{\problemsol}{\smallskip\stepcounter{problemcounter}\Item{\solutionmark\bfseries\arabic{problemcounter}.\ }}
\newcommand{\ppart}{\stepcounter{partcounter}\IItem{\bfseries\alph{partcounter})\ }}
\newcommand{\pparts}[1]{\vskip\parskip
   \halign{\hskip40pt\stepcounter{partcounter}\llap{\bfseries\alph{partcounter})\ }$##$\qquad\hfil&&
            \hskip30pt\stepcounter{partcounter}\llap{\bfseries\alph{partcounter})\ }$##$\qquad\hfil\cr
   #1\crcr}}
%\tparts is same as \pparts, but doesn't use math mode
\newcommand{\tparts}[1]{\vskip\parskip
   \halign{\hskip40pt\stepcounter{partcounter}\llap{\bfseries\alph{partcounter})\ }##\qquad\hfil&&
            \hskip30pt\stepcounter{partcounter}\llap{\bfseries\alph{partcounter})\ }##\qquad\hfil\cr
   #1\crcr}}
\newenvironment{exercises}
   {\setcounter{problemcounter}{0}
    \bigbreak\medbreak
    \leftline{\bfseries\large Exercises}
    \medskip
    \small}
   {}
\newcounter{solutioncounter}
\newcounter{spartcounter}[solutioncounter]
\newenvironment{solutions}
   {\setcounter{solutioncounter}{0}
    \bigbreak\medbreak
    %\leftline{\bfseries\large Selected Solutions}
    \medskip
    \small}
   {}

\newcommand{\solution}{\smallskip\stepcounter{solutioncounter}\Item{\bfseries\arabic{solutioncounter}.\ }}
\newcommand{\spart}{\stepcounter{spartcounter}\IItem{\bfseries\alph{spartcounter})\ }}
\newcommand{\sparts}[1]{\vskip\parskip
   \halign{\hskip40pt\stepcounter{spartcounter}\llap{\bfseries\alph{spartcounter})\ }$##$\qquad\hfil&&
            \hskip30pt\stepcounter{spartcounter}\llap{\bfseries\alph{spartcounter})\ }$##$\qquad\hfil\cr
   #1\crcr}}
%\tparts is same as \pparts, but doesn't use math mode
\newcommand{\stparts}[1]{\vskip\parskip
   \halign{\hskip40pt\stepcounter{spartcounter}\llap{\bfseries\alph{spartcounter})\ }##\qquad\hfil&&
            \hskip30pt\stepcounter{spartcounter}\llap{\bfseries\alph{spartcounter})\ }##\qquad\hfil\cr
   #1\crcr}}

\newcommand{\AND}{\wedge}
\newcommand{\OR}{\vee}
\newcommand{\NOT}{\lnot}
\newcommand{\IMP}{\rightarrow}
\newcommand{\IFF}{\leftrightarrow}
\newcommand{\XOR}{\oplus}
\newcommand{\T}{\mathbb{T}}
\newcommand{\F}{\mathbb{F}}
\newcommand{\LOGIMP}{\Longrightarrow}

%For Chapter 3
\newcommand{\N}{{\mathbb N}}
\newcommand{\Z}{{\mathbb Z}}
\newcommand{\Q}{{\mathbb Q}}
\newcommand{\R}{{\mathbb R}}
\newcommand{\Zpos}{{{\mathbb Z}^+}}
\newcommand{\SUB}{\subseteq}
\newcommand{\SUP}{\supseteq}
\newcommand{\PSUB}{\varsubsetneq}
\newcommand{\PSUBALT}{\subset}
\newcommand{\PSUP}{\varsupsetneq}
\newcommand{\SETDIFF}{\smallsetminus}   % the set difference operator.
\newcommand{\SETDIFFALT}{-}   % Alternative
\newcommand{\POW}{{\mathscr P}}   % For the power set of a set
\newcommand{\st}{\,|\,}   % for the | in sets { x | P(x) }

\begin{document}

\section*{Solutions 2.1}%~\ref{S-logic-1}}

\begin{solutions}
\setcounter{solutioncounter}{0}
\solution
\begin{tabular}{cc | c}
	\toprule
	$p$& $q$& $p \OR q$\\
	\midrule
	\strut
	0& 0& 0\\
	0& 1& 1\\
	1& 0& 1\\
	1& 1& 1\\
	\bottomrule
\end{tabular}
\hfill
\begin{tabular}{cc | c}
	\toprule
	$p$& $q$& $p \AND q$\\
	\midrule
	\strut
	0& 0& 0\\
	0& 1& 0\\
	1& 0& 0\\
	1& 1& 1\\
	\bottomrule
\end{tabular}
\hfill
\begin{tabular}{c | c}
	\toprule
	$p$& $\NOT p$\\
	\midrule
	\strut
	0& 1\\
	1& 0\\
	\bottomrule
\end{tabular}

\setcounter{solutioncounter}{1}
\solution 
\spart
\begin{tabular}{cc | ccc}
        \toprule
        $p$& $q$& 
        $\overbrace{p\IMP q}^\text{A}$ & 
		$\overbrace{p\AND (A)}^\text{B}$ &
        $(B)\IMP q$\\
        \midrule
        \strut 
		0& 0&  1& 0& 1\\
        0& 1&  1& 0& 1\\
        1& 0&  0& 0& 1\\
        1& 1&  1& 1& 1\\
        \bottomrule
     \end{tabular}
 
\hfill\\
Since the proposition is always true, the proposition is a tautology. 

\spart
\begin{tabular}{ccc | ccccc}
	\toprule
	$p$& $q$& $r$ &
	$\overbrace{p\IMP q}^\text{A}$ & 
	$\overbrace{q \IMP r}^\text{B}$ &
	$\overbrace{(A)\AND (B)}^\text{C}$ &
	$\overbrace{p \IMP r}^\text{D}$ &
	$(C) \IMP (D)$
	\\
	\midrule
	\strut 
	0& 0&  0& 1& 1 & 1 & 1 & 1\\
	0& 0&  1& 1& 1 & 1 & 1 & 1\\
 	0& 1&  0& 1& 0 & 0 & 1 & 1\\
	0& 1&  1& 1& 1 & 1 & 1 & 1\\
	1& 0&  0& 0& 1 & 0 & 0 & 1\\
	1& 0&  1& 0& 1 & 0 & 1 & 1\\
	1& 1&  0& 1& 0 & 0 & 0 & 1\\
	1& 1&  1& 1& 1 & 1 & 1 & 1\\
	\bottomrule
\end{tabular}

\hfill\\
Since the proposition is always true, the proposition is a tautology. 

\spart
\begin{tabular}{c | c}
	\toprule
	$p$& 
	$p\AND \NOT p$ \\
	\midrule
	\strut 
	0&0\\
	1&0\\
	\bottomrule
\end{tabular}

\hfill\\
Since the proposition is always false, the proposition is a contradiction.
\spart
\begin{tabular}{cc | ccc}
	\toprule
	$p$& $q$& 
	$\overbrace{p\OR q}^\text{A}$ &
	$\overbrace{p\AND q}^\text{B}$ &
	$(A)\IMP (B)$\\
	\midrule
	\strut 
	0& 0&  0& 0& 1\\
	0& 1&  1& 0& 0\\
	1& 0&  1& 0& 0\\
	1& 1&  1& 1& 1\\
	\bottomrule
\end{tabular}

\hfill\\
Since the proposition is sometimes true and sometimes false, the proposition is a contingency. 
\spart
\begin{tabular}{c | c}
	\toprule
	$p$& 
	$p\OR \NOT p$ \\
	\midrule
	\strut 
	0&1\\
	1&1\\
	\bottomrule
\end{tabular}

\hfill\\
Since the proposition is always true, the proposition is a tautology. 
\spart
\begin{tabular}{cc | ccc}
	\toprule
	$p$& $q$& 
	$\overbrace{p\AND q}^\text{A}$ &
	$\overbrace{p\OR q}^\text{B}$ &
	$(A)\IMP (B)$\\
	\midrule
	\strut 
	0& 0&  0& 0& 1\\
	0& 1&  0& 1& 1\\
	1& 0&  0& 1& 1\\
	1& 1&  1& 1& 1\\
	\bottomrule
\end{tabular}

\hfill\\
Since the proposition is always true, the proposition is a tautology.

\setcounter{solutioncounter}{2}
\solution
We will compare the truth tables of the subquestions to that of $p \IFF q$: 
\begin{tabular}{cc | c}
	\toprule
	$p \IFF q$\\
	\midrule
	\strut
	0& 0& 1\\
	0& 1& 0\\
	1& 0& 0\\
	1& 1& 1\\
	\bottomrule

\end{tabular}
\spart
\begin{tabular}{cc | ccc}
	\toprule
	$p$& $q$&
	$\overbrace{p \IMP q}^\text{A}$&
	$\overbrace{q \IMP p}^\text{B}$&
	$A \AND B$\\
	\midrule
	\strut
	0& 0& 1& 1& 1\\
	0& 1& 1& 0& 0\\
	1& 0& 0& 1& 0\\
	1& 1& 1& 1& 1\\
	\bottomrule
\end{tabular}

\spart
\begin{tabular}{cc | ccc}
	\toprule
	$p$& $q$&
	$\overbrace{\NOT p}^\text{A}$&
	$\overbrace{\NOT q}^\text{B}$&
	$A \IFF B$\\
	\midrule
	\strut
	0& 0& 1& 1& 1\\
	0& 1& 1& 0& 0\\
	1& 0& 0& 1& 0\\
	1& 1& 0& 0& 1\\
	\bottomrule
\end{tabular}

\spart
\begin{tabular}{cc | ccccc}
	\toprule
	$p$& $q$&
	$\overbrace{p \IMP q}^\text{A}$&
	$\overbrace{\NOT p}^\text{B}$&
	$\overbrace{\NOT q}^\text{C}$&
	$\overbrace{B \IMP C}^\text{D}$&
	$A \AND D$\\
	\midrule
	\strut
	0& 0& 1& 1& 1& 1& 1\\
	0& 1& 1& 1& 0& 0& 0\\
	1& 0& 0& 0& 1& 1& 0\\
	1& 1& 1& 0& 0& 1& 1\\
	\bottomrule
\end{tabular}

\spart
\begin{tabular}{cc | cc}
	\toprule
	$p$& $q$&
	$\overbrace{p \XOR q}^\text{A}$&
	$\NOT A$\\
	\midrule
	\strut
	0& 0& 0& 1\\
	0& 1& 1& 0\\
	1& 0& 1& 0\\
	1& 1& 0& 1\\
	\bottomrule
\end{tabular}

\setcounter{solutioncounter}{3}
\solution 
\begin{tabular}{ccc | cccc}
	\toprule
	$p$& $q$& $r$ &
	$\overbrace{p\IMP q}^\text{A}$ & 
	$\overbrace{A \IMP r}^\text{B}$ &
	$\overbrace{q \IMP r}^\text{C}$ &
	$\overbrace{p \IMP C}^\text{D}$
	\\
	\midrule
	\strut 
	0& 0&  0& 1& 0 & 1 & 1\\
	0& 0&  1& 1& 1 & 1 & 1\\
	0& 1&  0& 1& 0 & 0 & 1\\
	0& 1&  1& 1& 1 & 1 & 1\\
	1& 0&  0& 0& 1 & 1 & 1\\
	1& 0&  1& 0& 1 & 1 & 1\\
	1& 1&  0& 1& 0 & 0 & 0\\
	1& 1&  1& 1& 1 & 1 & 1\\
	\bottomrule
\end{tabular}

\hfill\\
Since the truth tables for the expressions (columns B and D) are different, they are not equivalent. As our counterexample take for instance: $p=q=r=0$. Thus $\IMP$ is not associative. \\
What about $\IFF$?

\setcounter{solutioncounter}{4}
\solution
\spart
$p \OR q$ 
\spart
$\NOT p \IMP q$

\setcounter{solutioncounter}{5}
\solution
The four propositions are:
%\stparts{
\begin{enumerate}
	\item[\textbf{a)}] Galileo was not falsely accused and the Earth is the centre of the universe. %\cr
	\item[\textbf{b)}] If the Earth moves then the Earth is not the centre of the universe. %\cr
	\item[\textbf{c)}] The Earth moves if and only if the Earth is not the centre of the universe. %\cr
	\item[\textbf{d)}] If the Earth moves the Galileo was falsely accused, but if the Earth is the centre of the universe then Galileo was not falsely accused.
\end{enumerate}
%}

\setcounter{solutioncounter}{6}
\solution
\spart
\begin{enumerate}
	\item[\textbf{Converse}] If Sinterklaas brings you toys, you are good.
	\item[\textbf{Contrapositive}] If Sinterklaas does not bring you toys, you are not good.
\end{enumerate}
\spart
\begin{enumerate}
	\item[\textbf{Converse}] If you need extra postage, then the package weighs more than one kilo.
	\item[\textbf{Contrapositive}] If you do not need extra postage, then the package does not weigh more than one kilo.
\end{enumerate}
\spart
\begin{enumerate}
	\item[\textbf{Converse}] If I don't eat courgette, I have a choice.
	\item[\textbf{Contrapositive}] If I eat courgette, I don't have a choice.
\end{enumerate}

\setcounter{solutioncounter}{7}
\solution
\spart
The only card that satisfies this is the ten of hearts.
\spart
An ordinary deck has 4 cards that satisfy the condition of being a ten, and 13 cards that satisfy the condition of being a heart, the ten of hearts has been counted twice. So the total the amount of cards that satisfy all conditions is 16.
\spart
All cards that are not a ten will satisfy this condition, as well as the cards that are a ten and a heart, which is only the ten of hearts. So only three cards do not satisfy this condition, which are the ten of diamonds, ten of spades, and ten of clubs.
\spart
It's easier to reason about the cards that do not satisfy the condition. All cards that are a ten and not a heart do not satisfy the condition, as well as all cards that are a heart and not a ten. There are 3 cards that are a ten and not a heart (ten of diamonds, ten of spades, and ten of clubs). There are 12 cards that are a heart and not a ten. So the total amount of cards that do not satisfy this condition is 15, which means that there are 37 cards that satisfy the condition.

\setcounter{solutioncounter}{8}
\solution
A $\ast$ is used to indicate the step at which we can stop rewriting, as the equation will use $\downarrow$ or other operators shown previously.
\begin{enumerate}
	\item \begin{align*}
	\NOT p & \equiv (\NOT p \AND \NOT p)\\
	& \equiv \NOT (p \OR p)\\
	& \equiv p \downarrow p
	\end{align*}
	\item \begin{align*}
	p \AND q & \equiv \NOT (\NOT p \OR \NOT q)\\
	& \equiv \NOT p \downarrow \NOT q \tag{$\ast$}\\
	& \equiv (p \downarrow p) \downarrow (q \downarrow q)
	\end{align*}
	\item \begin{align*}
	p \OR q & \equiv \NOT \NOT (p \OR q)\\
	& \equiv \NOT (p \downarrow q) \tag{$\ast$}\\
	& \equiv (p \downarrow q) \downarrow (p \downarrow q)
	\end{align*}
	\item \begin{align*}
	p \IMP q & \equiv \NOT p \OR q \tag{$\ast$}\\
	& \equiv (\NOT p \downarrow q) \downarrow (\NOT p \downarrow q) \\
	& \equiv ((p \downarrow p) \downarrow q) \downarrow ((p \downarrow p) \downarrow q)
	\end{align*}
	\item \begin{align*}
	p \IFF q & \equiv (p \IMP q) \AND (q \IMP p) \tag{$\ast$}\\
	& \equiv (((p \downarrow p) \downarrow q) \downarrow ((p \downarrow p) \downarrow q)) \AND (((q \downarrow q) \downarrow p) \downarrow ((q \downarrow q) \downarrow q)) \\
	& \equiv (((((p \downarrow p) \downarrow q) \downarrow ((p \downarrow p) \downarrow q)) \downarrow (((p \downarrow p) \downarrow q) \downarrow ((p \downarrow p) \downarrow q))) \notag\\ 
	& \quad \downarrow ((((q \downarrow q) \downarrow p) \downarrow ((q \downarrow q) \downarrow p)) \downarrow (((q \downarrow q) \downarrow p) \downarrow ((q \downarrow q) \downarrow p))))
	\end{align*}
	\item \begin{align*}
	p \XOR q & \equiv (p \AND \NOT q) \OR (\NOT p \AND q) \tag{$\ast$}\\
	& \equiv (p \AND (q \downarrow q)) \OR ((p \downarrow p) \AND q)\\
	& \equiv ((p \downarrow p) \downarrow ((q \downarrow q) \downarrow (q \downarrow q))) \OR (((p \downarrow p) \downarrow (p \downarrow p)) \downarrow (q \downarrow q))\\
	& \equiv ((((p \downarrow p) \downarrow ((q \downarrow q) \downarrow (q \downarrow q))) \downarrow (((p \downarrow p) \downarrow (p \downarrow p)) \downarrow (q \downarrow q))) \notag\\
	& \quad \downarrow (((p \downarrow p) \downarrow ((q \downarrow q) \downarrow (q \downarrow q))) \downarrow (((p \downarrow p) \downarrow (p \downarrow p)) \downarrow (q \downarrow q))))
	\end{align*}
\end{enumerate}

\setcounter{solutioncounter}{9}
\solution
\spart
A truth table for a formula containing two unique atoms will have 4 rows. Each of these 4 rows could have either a $0$ or a $1$ as result. Which means that there will be $2^4$ unique truth tables for a formula containing 2 unique atoms.
\spart
\begin{tabular}{cc | cccccccc}
	\toprule
	$p$& $q$&
	$p \AND \NOT p$&
	$p \AND q$&
	$p \AND \NOT q$&
	$p$&
	$\NOT p \AND q$&
	$q$&
	$\NOT(p \IFF q)$&
	$p \OR q$
	\\
	\midrule
	\strut 
	0& 0&  0& 0& 0& 0& 0& 0& 0& 0\\
	0& 1&  0& 0& 0& 0& 1& 1& 1& 1\\
	1& 0&  0& 0& 1& 1& 0& 0& 1& 1\\
	1& 1&  0& 1& 0& 1& 0& 1& 0& 1\\
	\bottomrule
\end{tabular}
\\
\begin{tabular}{cc | cccccccc}
	\toprule
	$p$& $q$&
	$\NOT p \AND \NOT q$&
	$p \IFF q$&
	$\NOT q$&
	$q \IMP p$&
	$\NOT p$&
	$p \IMP q$&
	$\NOT p \OR \NOT q $&
	$p \OR \NOT p$
	\\
	\midrule
	\strut 
	0& 0&  1& 1& 1& 1& 1& 1& 1& 1\\
	0& 1&  0& 0& 0& 0& 1& 1& 1& 1\\
	1& 0&  0& 0& 1& 1& 0& 0& 1& 1\\
	1& 1&  0& 1& 0& 1& 0& 1& 0& 1\\
	\bottomrule
\end{tabular}
\spart
Given that we can create a formula for each of the possible truth table using these 5 operators. Furthermore, we know that there are only $2^{2^{n}}$ possible truth tables, where $n$ is the number of unique atoms. Given that every one of these truth tables has a corresponding formula using only the 5 operators, which we denote as $f_i$, where $1 \leq i \leq 2^{2^{n}} - 1$. Now take a formula, $f'$, that uses the same unique atoms, but not necessarily the same operators from the set of 5 operators stated in the description. The truth table for the formula $f'$ will be equal to one of the $2^{2^{n}}$ possible truth tables, which also means that there is a $f_i$ that is equivalent to $f'$, so we can rewrite $f'$


\end{solutions}

\section*{Solutions~2.2}%\ref{sec:boolean-algebra}}


\begin{solutions}
	\setcounter{solutioncounter}{3}
	\solution It is not. Take for example $p=q=0$. In this case $\NOT(p \IFF q)$ is false, but $(\NOT p) \IFF (\NOT q)$ is true. Try to see if you can find a more simplified expression that $(\NOT p) \IFF (\NOT q)$ is equivalent to.
	\setcounter{solutioncounter}{9}
	\solution Verify the following answers with truth tables yourself!
	\sparts{
		q \IMP p &
		\F &
		\NOT p \cr
		\NOT p \AND q &
		\T &
		q
	}
	\solution When translating try make the English sentences flow a bit without adding in more constraints. For example by using the word `but' rather than `and' in two of the examples below.
	\spart
		It is not sunny or it is not cold.
	\spart
		I will have neither stroopwafels nor appeltaart.
	\spart
		It is Tuesday today, but this is not Belgium.
	\spart
		You passed the final exam, but you did not pass the course.
	
\end{solutions}

\section*{Solutions~2.3}%\ref{S-logic-4}}

\begin{solutions}
	\solution
	\sparts{
		\exists x (P(x)) & \forall x(\NOT P(x)\OR \NOT Q(x))\cr
		\exists z(P(z) \AND \NOT Q(z))&
		(\exists x (\NOT P(x)))\OR (\exists y (\NOT Q(y))) \cr
		\exists x \forall y \NOT P(x,y)&
		\forall x (\NOT R(x)\OR \exists y \NOT S(x,y))\cr
		\forall y((\NOT P(y)\AND Q(y)) \OR (P(y) \AND \NOT Q(y)))&
		\exists x (P(x) \AND (\forall y \NOT Q(x,y)))\cr
	}
\setcounter{solutioncounter}{7}
\solution We use the predicates $\textit{Ball}(x)$ for $x$ is a ball and $\textit{Have}(x,y)$ for $x$ must have $y$. We also use a constant $\textit{you}$ to represent you.
\[
	\exists x (\textit{Ball}(x) \AND \forall y (x \neq y \IMP \NOT \textit{Ball}(y)) \AND \textit{Have}(\textit{you}, x))
\]
\setcounter{solutioncounter}{12}
\solution Using the predicates $\textit{Person}(x)$ for $x$ is a person, $\textit{Question}(x)$ for $x$ is a question, $\textit{Answer}(x)$ for $x$ is an answer, and $\textit{Has}(x,y)$ for $x$ has $y$.
Two different interpretations are:
\[
	\exists x (\textit{Person}(x) \AND \forall y (\textit{Question}(y) \IMP (\exists z (\textit{Answer}(z) \AND \textit{Has}(x,z)))))
\]
In other words, there is a single person who has the answers to all questions. 
\[
	\forall x (\textit{Question}(x) \IMP \exists y \exists z (\textit{Person}(y) \AND \textit{Answer}(z) \AND \textit{Has}(y,z)))
\]
In other words, every question has an answer and some person knows it (but different people might know the answer to different questions).
\end{solutions}

\section*{Solutions~2.5}%\ref{deductions}}

\begin{solutions}
	\solution In order to verify the validity of \textit{modus tollens}, we need to verify that $((p \IMP q)\AND \NOT q) \IMP \NOT p$ is a tautology. This can be done by constructing the following truth table. The truth table shows that the conclusion is always true, and thus verifies that the \textit{modus tollens} is valid.\\
	
	\begin{tabular}{cc | ccc}
		\toprule
		$p$& $q$& $\overbrace{p\IMP q}^\text{A}$& $\overbrace{A \AND \NOT q}^\text{B}$& $B \IMP \NOT p$ \\
		\midrule
		\strut
		0& 0& 1& 1& 1\\
		0& 1& 1& 0& 1\\
		1& 0& 0& 0& 1\\
		1& 1& 1& 0& 1\\
		\bottomrule
	\end{tabular} \\
	
	To verify the validity of the Law of Syllogism, construct another truth table to show that $((p \IMP q) \AND (q \IMP r)) \IMP (p \IMP r)$ is also a tautology. 
	
	\solution Note that the following answers are example arguments, and that there exist many other valid answers. 
	
	\begin{itemize}
		\item Since it isn't day, it must be night.
		\item I have bread and I have cheese on top, so I have a cheese sandwich. 
		\item I have a cheese sandwich, so I have cheese. 
		\item Since I sing all the time, I am always singing or talking. 
	\end{itemize}
	

  \solution
  For both arguments when $p$ is false and $q$ is true the premises hold but the
  conclusion does not.

  \begin{itemize}
    \item When it rains the ground is wet. The ground is wet. Therefore it rains.
    \item When I am on a boat i am not on land. I am not on a boat. Therefore I
      am on land.
  \end{itemize}



  \solution
	For this set of solutions, remember that you a slightly less formal method would be to use a truth table to prove the validity of arguments in propositional logic. If you can show that in all rows where all premises are true, the conclusion is also true, then the argument must be valid!
  \spart
    Invalid. A counterexample for this argument is when $p$ and $q$ are false and $s$ is
    true.
    \spart Valid.
    \begin{align}
      p\wedge q && \text{(premise)}\\
      q && \text{(from 1)}\\
      q\rightarrow (r\vee s) && \text{(premise)}\\
      r\vee s && \text{(from 2 and 3 by \textit{modus ponens})}\\
      \lnot r && \text{(premise)}\\
      s && \text{(from 4 and 5)}
    \end{align}
    \spart Valid.
    \setcounter{equation}{0}
    \begin{align}
      p\vee q && \text{(premise)}\\
      \lnot p  && \text{(premise)}\\
      q && \text{(from 1 and 2)}\\
      q\rightarrow (r\wedge s)&&\text{(premise)}\\
      r\wedge s&&\text{(from 3 and 4 by \textit{modus ponens})}\\
      s&&\text{(from 5)}
    \end{align}
    \spart Valid.
    \setcounter{equation}{0}
    \begin{align}
      \lnot(q\vee t)&&\text{(premise)}\\
      \lnot q\wedge\lnot t&&\text{(from 1 by \textit{De Morgan})}\\
      \lnot t&&\text{(from 2)}\\
      (\lnot p)\rightarrow t&&\text{(premise)}\\
      \lnot(\lnot p)&&\text{(from 3 and 4 by \textit{modus tollens})}\\
      p&&\text{(from 5)}
    \end{align}
    \spart Invalid. A counterexample for this argument is when $p$, $r$ and $s$
    are true and $q$ is false.
    \spart Valid.
    \setcounter{equation}{0}
    \begin{align}
      p&&\text{(premise)}\\
      p\rightarrow(t\rightarrow s)&&\text{(premise)}\\
      t\rightarrow s&&\text{(from 1 and 2 by \textit{modus ponens})}\\
      q\rightarrow t&&\text{(premise)}\\
      q\rightarrow s&&\text{(from 3 and 4 by \textit{Law of Syllogism})}
    \end{align}
 
 	\solution
	\setcounter{solutioncounter}{4}
	\spart 
	Let $s$ = "today is a Sunday", $r$ = "it rains today" and $s$ = "it snows today". \\
	\begin{tabular}{l}
		$s \IMP (r \OR s)$\\
		$s$ \\
		$\NOT r$\\
		\hline
		$\therefore s$
	\end{tabular} \\
	This argument is valid. 
	
	\spart
	Let $h$ = "there is herring on the pizza", $n$ = "Jack doesn't eat pizza", $a$ = "Jack is angry". \\
	\begin{tabular}{l}
		$h \IMP n$\\
		$n \IMP a$ \\
		$a$\\
		\hline
		$\therefore h$
	\end{tabular} \\
	The argument is invalid since we can't deduce anything from the fact that Jack is angry, there might be many more reasons for Jack to get angry. \\
	Note that this exercise becomes harder when we translate to predicate logic instead of propositional logic. For predicate logic, "a" pizza is translated differently than "the" pizza.\\ 
	This exercise in predicate logic would become: \\
	Let $P(x)$ mean that $x$ is a pizza, let $H(x)$ mean that $x$ is a herring, let $On(x,y)$ mean that $x$ is on $y$, let $E(x,y)$ mean $x$ eats $y$ and finally let $A(x)$ mean that $x$ is angry. Let j = Jack. \\
	\begin{tabular}{l}
		$P(a)$ \\
		$\forall x((H(x) \AND On(x,a)) \IMP \NOT E(j,a))$ \\
		$\NOT \exists x((P(x) \AND E(j,x)) \IMP A(j))$ \\
		$A(j)$ \\
		\hline
		$\therefore \exists x (H(x) \AND On(x,a))$
	\end{tabular} \\
	As an exercise for yourself, try to translate part c to predicate logic as well as propositional logic.
	
	\spart
	Let $a$ = "it is 8:00", $l$ = "Jane studies at the library" and $h$ = "Jane works at home". \\
	\begin{tabular}{l}
		$a \IMP (l \OR h)$\\
		$a$ \\
		$\NOT l$\\
		\hline
		$\therefore h$
	\end{tabular} \\
	This argument is valid.
	
	
	
\end{solutions}

\section*{Solutions~3.2}%\ref{S-proof}}

\begin{solutions}
\setcounter{solutioncounter}{4}
\solution
The claim is: $(r \mid s) \wedge (s \mid t) \to r \mid t$.
\begin{proof}
	We need to show something is true for all integers $r, s, t$, so take arbitrary integers $k, m, n$ such that: $k \mid
	m$, $m \mid n$. \\
	Now we need to prove that $k \mid n$ holds.\\
	Since $k \mid m$, we know that $m = ak$ for some integer $a$. Similarly $n = bm$ for some integer $b$. \\
	Thus $n = bm = bak = ck$ with integer $c= ba$. \\
	Thus $k \mid n$.\\
	Since $k,m,n$ were arbitrary, this holds for all integers.
\end{proof}

\end{solutions}

\section*{Solutions~3.3}%\ref{S-proof-by-contradiction}}

\begin{solutions}
	
\setcounter{solutioncounter}{0}
\solution
\begin{proof}
	Assume that all of the numbers $a_i$ are smaller than or equal to 10. Since the maximum value of each $a_i$ is 10, the maximum value of $a_1+a_2+...+a_{10}$ is now 100. However we asssumed that the summation was strictly larger than 100. This contradiction means our assumption that all numbers are smaller than or equal to 10 must have been false and so at least one $a_i$ must be greater than 10.
\end{proof}
	
	
\setcounter{solutioncounter}{1}
\solution
\begin{itemize}
	\item \verb|a)|
	\begin{proof}
		Assume that there exists an integer that is odd whose square is even, i.e. $\exists n \in \mathbb{Z} ((2 \mid n^2) \land (2 \nmid n)) \equiv \exists n \in \mathbb{Z} ((2 \nmid n) \land (2 \mid n^2))$. Take an arbitrary odd integer $m = 2p - 1, p \in \mathbb{Z}$. By taking its square, we get
		\begin{align*}
			m^2 = (2k-1)^2
			 = 4k^2 - 4k + 1
			 = 2(2k^2 - 2k) + 1
			 = 2a + 1, a \in \mathbb{Z}
		\end{align*}
		This means that the square is again odd. Because the integer we have taken was arbitrary, this is the case for all odd integers. From this follows a contradiction: in our assumptions, we stated that there must exist an integer that is odd whose square is even, which can never happen. For this reason, the assumption is false, and the original claim must be valid.
	\end{proof}

	\item \verb|b)|
	\begin{proof}
		Assume $\sqrt{2}$ to be rational. This means that $\sqrt{2}$ can be written as 
		\begin{align*}
			\sqrt{2} = \frac{a}{b}, a, b \in \mathbb{Z}
		\end{align*}
		where a and b do not have any common divisors (= the fraction cannot be simplified). Then:
		\begin{align}
			2 = \frac{a^2}{b^2}\\
			2b^2 = a^2
		\end{align}
		This implies that $a^2$ is even. From \verb|a)|, we know that if $a^2$ is even, then a is also even.\\
		A can now be written as $a = 2k, k \in \mathbb{Z}$.
		\begin{align}
			2b^2 = a^2 = (2k)^2 = 4k^2\\
			2b^2 = 4k^2\\
			b^2 = 2k^2
		\end{align}
		Consequently, $b^2$ is even so b is even. However, because $\frac{a}{b}$ was supposed to not have any common divisors and a and b are both even (must have 2 as a common divisor), we derived a contradiction. This disproves our assumption and as a consequence, $\sqrt{2}$ must be irrational.\\
		
		\emph{Note: in this case, doing the proof by contradiction and taking the inverse of the statement helped, because we switched from working with an irrational number to working with fractions of integers.}
	\end{proof}

	\item \verb|c)|
	\begin{proof}
		Assume that the sum of a rational and an irrational number is rational. Then:
		\begin{align}
			r = \frac{a}{b}, a, b \in \mathbb{Z}, b \ne 0\\
			r + x = \frac{c}{d}, c, d \in \mathbb{Z}, d \ne 0
		\end{align}
		Where $x \in \mathbb{R} - \mathbb{Z}$ (i.e., is irrational).
		These fractions cannot be simplified. Now:
		\begin{align}
			\frac{a}{b} + x &= \frac{c}{d}\\
			x &= \frac{c}{d} - \frac{a}{b}\\
			&= \frac{cb - ad}{db}, db \ne 0
		\end{align}
		The term $cd - ab$ is an integer, and so is $db$. This would mean that x is a fraction of 2 integers - in other words, x is rational. Because we assumed x to be irrational, this is a contradiction. We can therefore conclude that the sum of a rational and irrational number must be irrational.\\
		
		\emph{Note: in this case, doing the proof by contradiction and taking the inverse of the statement helped, because we switched from working with an irrational number to working with fractions of integers.}
	\end{proof}

	\item \verb|d)|
	\begin{proof}
		Assume that rx is rational. Then:
		\begin{align}
			r = \frac{a}{b}, a, b \in \mathbb{Z}, a, b \ne 0\\
			rx = \frac{c}{d}, c, d \in \mathbb{Z}, c, d \ne 0
		\end{align}
		Where $x \in \mathbb{R} - \mathbb{Z}$ (i.e., is irrational).
		These fractions cannot be simplified. Then:
		\begin{align}
			rx &= \frac{c}{d} = \frac{xa}{b}\\
			cb &= xad\\
			x &= \frac{cb}{ad}, a, b, c, d \ne 0
		\end{align}
		The term $cd$ is a nonzero integer, and so is $ad$. This would mean that x is a fraction of 2 integers - in other words, x is rational. Because we assumed x to be irrational, this is a contradiction. We can therefore conclude that the product of a rational and irrational number must be irrational.\\
		
		\emph{Note: in this case, doing the proof by contradiction and taking the inverse of the statement helped, because we switched from working with an irrational number to working with fractions of integers.}
	\end{proof}

	\item \verb|e)|
	\begin{proof}
		Assume that r and r + x are rational and x is irrational. Then:
		\begin{align}
			r = \frac{a}{b}, a, b \in \mathbb{Z}, a, b \ne 0\\
			r + x = \frac{c}{d}, c, d \in \mathbb{Z}, c, d \ne 0
		\end{align}
		Where $x \in \mathbb{R} - \mathbb{Z}$ (i.e., is irrational).
		Now, it is important to realise that x cannot be expressed as a rational number, i.e. $\neg\exists i, j$ such that $(x = \frac{i}{j})$. Now:
		\begin{align}
			\frac{a}{b} + x = \frac{c}{d}\\
			x = \frac{c}{d} - \frac{a}{b}\\
			x = \frac{cb - ad}{db}
		\end{align}
		The term $cb - ad$ is a nonzero integer, and so is $db$. This would mean that x is a fraction of 2 integers - in other words, x is rational. Because we assumed x to be irrational, this is a contradiction. We can therefore conclude that if r and r + x are both rational, then x is rational.\\
		
		\emph{Note: here, the proof by contradiction did not help as much. Going from rational to irrational does not bring any improvement, because we only know how \strong{not} to write an irrational number.}
	\end{proof}
	
\end{itemize}
	
	
\setcounter{solutioncounter}{2}
\solution
\begin{proof}
	Assume that every hole has at most one pigeon in it.
	This means that there are $< k$ pigeons in total. Since $n > k$ this forms a contradiction. Therefore our assumption
	that every hole has at most one pigeon is incorrect and there must be at least one hole that has two or more pigeons.
\end{proof}
(Take care to flip the quantifiers correctly when doing a proof by contradiction! $\neg \forall x(\dots)$ becomes $\exists
x(\neg \dots)$.)

\end{solutions}

\section*{Solutions~3.5}%\ref{S-logic-4}}

\begin{solutions}
\setcounter{solutioncounter}{7}
\solution 
\spart $\sum\limits_{i=0}^{9} (2*i + 1)$
\spart $\sum\limits_{i=0}^{6} (\frac{1}{3^i})$
\spart $\sum\limits_{i=50}^{100} (i)$
\spart $\sum\limits_{i=1}^{10} (i^2)$
\spart $\sum\limits_{i=2}^{99} (\frac{1}{2^i})$
\end{solutions}

\section*{Solutions~3.7}%\ref{S-logic-4}}

\begin{solutions}
	
	\setcounter{solutioncounter}{0}
	\solution 
	
	\textit{Proof:} We prove this once again by induction. 
	
	\begin{itemize}
		\item Base Case: When $n = 0$ and when $n = 1$, the statement is clearly true, since $f_0 = 0 < 1 = 2^0$ and $f_1 = 1 < 2 = 2^1$, so $f_0 < 2^0$ and $f_1 < 2^1$.
		
		\item Inductive Case: If we now take an arbitrary $k$ such that $k \geq 2$, we assume that $f_n < 2^n$ for $n = k-1$ and $n = k-2$ holds. 
		To complete the induction, we need to show that $f_k < 2^k$ is also true: 
	\end{itemize}
	\begin{align*}
		f_k &= f_{k-1} + f_{k-2} \\
		&< 2^{k-1} + 2^{k-2}  &\textnormal{(inductive hypothesis)} \\
		&= \frac{2^{k}}{2} + \frac{2^{k}}{4} \\
		&= \frac{3}{4} \cdot 2^k \\
		&< 2^k
	\end{align*}	
	
		This shows that $f_k < 2^k$ is true, which completes the induction. 
		
	\solution 
	
	\textit{Proof:} We prove the equation using induction. 
	
	\begin{itemize}
		\item Base Case: When $n = 1$, $a_1 = 1\cdot2^{1-1}$ is true since both sides are equal to 1. 
		
		\item Inductive Case: Let $k \geq 1$ be an arbitrary integer. We asssume that $a_n = n2^{n-1}$ holds for $n = k-1$. 
		To complete the induction, we need to show that the equation also holds for $n = k$.
	\end{itemize}
	\begin{align*}
		a_k &= 2a_{k-1} + 2^{k-1} \\
		&= 2(k-1)2^{k-1-1} + 2^{k-1} &\text{(inductive hypothesis)} \\
		&= (k-1)2^{k-1} + 2^{k-1} \\
		&= ((k-1)+1)2^{k-1} \\
		&= k2^{k-1}
	\end{align*}	
	
	Which proves that the equation also holds for $n=k$. Thereby the induction is completed.  
	
\end{solutions}


\section*{Solutions~4.1}%\ref{S-sets-1}}

\begin{solutions}
	% 1
	\solution
		The possibilities are 2,3,4 and 5 different elements.
		\begin{itemize}
			\item 2: In the case that $a=b=c$. Then the set can be written as: \[\{a, a, \{a\}, \{a, a\}, \{a,a,a\}\}=\{a, \{a\}\}\]
			\item 3: In the case that $a=b\neq c$. Then the set can be written as: \[\{a, a, \{a\}, \{a, c\}, \{a,c\}\}=\{a, \{a\}, \{a, c\}\}\]
			\item 4: In the case that $a\neq b$ and $a=c$. Then the set can be written as: \[\{a, b, \{a\}, \{a, a\}, \{a,b,a\}\}=\{a, b, \{a\}, \{a, b\}\}\]
			\item 5: In the case that $a\neq b\neq c$. Then the set can be written as: \[\{a, b, \{a\}, \{a, c\}, \{a,b,c\}\}\].
		\end{itemize}

	% 2
	\solution
		\spart $A \cup B = \{a,b,c\}; A \cap B = \emptyset; A \SETDIFF B = \{a,b,c\}$
		\spart $A \cup B = \{1,2,3,4,5,6,8,10\}; A \cap B = \{2,4\}; A \SETDIFF B = \{1,3,5\}$
		\spart $A \cup B = \{a,b,c,d\}; A \cap B = \{a,b\}; A \SETDIFF B = \emptyset$
		\spart $A \cup B = \{a,b,\{a\},\{a,b\}\}; A \cap B = \{\{a,b\}\}; A \SETDIFF B = \{a,b\}$

	% 3
	\solution
		\spart \begin{tikzpicture}
			\begin{scope}[blend group=soft light]
				\fill[red!30!white] (0:1.2) circle (2);
				\fill[blue!30!white]  (180:1.2) circle (2);
			\end{scope}
			
			\node at (150:2.5)	{\Large\textbf{A}};
			\node at (30:2.5)	{\Large\textbf{B}};
			\node at (180:2.5)  {a};
			\node at (230:2)    {b};
			\node at (120:1.5)  {c};
		\end{tikzpicture}
		\spart \begin{tikzpicture}
		\begin{scope}[blend group=soft light]
		\fill[red!30!white] (0:1.2) circle (2);
		\fill[blue!30!white]  (180:1.2) circle (2);
		\end{scope}
		
		\node at (150:2.5)	{\Large\textbf{A}};
		\node at (30:2.5)	{\Large\textbf{B}};
		\node at (180:2.5)  {5};
		\node at (230:2)    {3};
		\node at (125:1.8)  {1};
		\node at (90:0.7)   {2};
		\node at (270:0.7)  {4};
		\node at (55:1.8)   {10};
		\node at (0:2.5)    {8};
		\node at (320:2)    {6};
		\end{tikzpicture}
		\spart \begin{tikzpicture}
		\begin{scope}[blend group=soft light]
		\fill[red!30!white] (0:1.2) circle (2);
		\fill[blue!30!white]  (180:1.2) circle (2);
		\end{scope}
		
		\node at (150:2.5)	{\Large\textbf{A}};
		\node at (30:2.5)	{\Large\textbf{B}};
		\node at (90:0.7)   {a};
		\node at (270:0.7)  {b};
		\node at (0:2.5)    {d};
		\node at (320:2)    {c};
		\end{tikzpicture}
		\spart\begin{tikzpicture}
		\begin{scope}[blend group=soft light]
		\fill[red!30!white] (0:1.2) circle (2);
		\fill[blue!30!white]  (180:1.2) circle (2);
		\end{scope}
		
		\node at (150:2.5)	{\Large\textbf{A}};
		\node at (30:2.5)	{\Large\textbf{B}};
		\node at (180:2.5)  {a};
		\node at (230:2)    {b};
		\node at (90:0)     {\{a,b\}};
		\node at (0:2.5)    {\{a\}};
		\end{tikzpicture}

	% 4
	\solution
		\spart $X\cap Y=\{5,6,7,8,9,10\}$
		\spart $X\cup Y=\mathbb{N}$
		\spart $X\SETDIFF Y=\{11,12,13,...\}$
		\spart $\mathbb{N}\SETDIFF Z=\{1,3,5,...\}$
		\spart $X\cap Z=\{6,8,10,...\}$
		\spart $Y\cap Z=\{0,2,4,6,8,10\}$
		\spart $Y\cup Z=\{0,1,2,3,4,5,6,7,8,9,10,12,14,16...\}$
		\spart $Z\SETDIFF \mathbb{N}=\emptyset$

	% 5
	\solution $\POW(\{1,2,3\})=\{\emptyset, \{1\}, \{2\}, \{3\}, \{1,2\}, \{1,3\}, \{2,3\}, \{1,2,3\}\}$

	% 6
	\solution
		\spart False. Although $b$ is part of the two sets that are elements of $A$, namely $\{b\}$ and $\{a,b\}$, the element $b$ itself is not a part of $A$.
		\spart False. In order for $\{a,b\}$ to be a subset of $A$, both $a$ and $b$ should be elements of $A$. Like we specified in \textbf{a)}, $b\notin A$, hence $\{a,b\}\NOT\SUB A$.
		\spart True. In order for $\{a,b\}$ to be a subset of $B$, both $a$ and $b$ should be elements of $B$, which they are.
		\spart False. Although $a$ and $b$ are both individual elements of $B$, the combined set $\{a,b\}$ is not.
		\spart False. Although $a$ and $\{b\}$ are both individual elements of $A$, the combined set $\{a,\{b\}\}$ is not.
		\spart True. $\{a, \{b\}\}$ is an element of $B$.

	% 7
	\solution Yes, this is possible.
	\[\POW(\POW(\emptyset))=\POW(\{\emptyset\})=\{\emptyset, \{\emptyset\}\}\]
	\begin{align*}
	\POW(\POW(\{a,b\}))=&\POW(\{\emptyset, \{a\}, \{b\}, \{a,b\}\})\\
	=&\{\emptyset,\\
	&\{\emptyset\},\{\{a\}\}, \{\{b\}\},\{\{a,b\}\},\\
	&\{\emptyset, \{a\}\}, \{\emptyset, \{b\}\}, \{\emptyset, \{a,b\}\}, \{\{a\}, \{b\}\}, \{\{a\}, \{a,b\}\}, \{\{b\}, \{a,b\}\},\\
	& \{\emptyset, \{a\}, \{b\}\}, \{\emptyset, \{a\}, \{a,b\}\},
	\{\emptyset, \{b\}, \{a,b\}\}, \{\{a\}, \{b\}, \{a,b\}\},\\
	&\{\emptyset, \{a\}, \{b\}, \{a,b\}\}\}
	\end{align*}

	% 8
	\solution The sentence "“She likes dogs that are small, cuddly, and cute" talks about dogs that are both small, and cuddly, and cute. Hence, the dogs she likes has to be in the set of small dogs, the set of cuddly dogs and the set of cute dogs. Therefore, the set of dogs she likes is the intersection of the three sets.\\
	On the other hand, the sentence "She likes dogs that are
	small, dogs that are cuddly, and dogs that are cute" talks about dogs that are either small, cuddly, or cute. Hence, the set of liked dogs is the union of these three sets.

	% 9
	\solution $A \cup A = A$ (remember that sets have no duplicates)\\
	$A \cap A = A$ (after all everything in $A$ that is also in $A$, is everything)\\
	$A \SETDIFF A = \emptyset$ (removing from $A$ everything that is in $A$ leaves us nothing)

	% 10
	\solution
	We know that $A\SUB B$. This means that each element of $A$ is in $B$ as well. Therefore, if we take $A\cup B$ we have a set which is equal to $B$, as all elements of $A$ are in $B$. With the same reasoning, $A\cap B$ is equal to $A$, as all elements of $A$ are in $B$. Moreover, because of this, $A\SETDIFF B$ renders the empty set, as no elements are left when removing all elements from $A$ which are in $B$ as well.

	% 11
	\solution
	\begin{proof}
		In order to prove that $C\SUB A\cap B\IFF (C\SUB A\land C\SUB B)$, we have to show that $C\SUB A\cap B\IMP (C\SUB A\land C\SUB B)$ and $(C\SUB A\land C\SUB B)\IMP C\SUB A\cap B$.
		\begin{itemize}
			\item $C\SUB A\cap B\IMP (C\SUB A\land C\SUB B)$:\\
				We assume that $C\SUB A\cap B$. Take an arbitrary element $x$ in $C$. Because $C\SUB A\cap B$, $x$ is in the intersection of $A$ and $B$. Because this is the case, $x$ is in both $A$ and in $B$. Because this holds for an arbitrary element in $C$, it holds for all elements in $C$. Therefore, all elements of $C$ are an element of both $A$ and $B$, and hence $C\SUB A\land C\SUB B$.
			\item $(C\SUB A\land C\SUB B)\IMP C\SUB A\cap B$:\\
				We assume that $C\SUB A$ and $C\SUB B$.Take an arbitrary element $x$ in $C$. Because $C\SUB A$ and $C\SUB B$, $x$ is both in $A$ and in $B$. Because of this, $x$ is in the intersection of $A$ and $B$ ($A\cap B$) as well. Because this holds for an arbitrary element in $C$, it holds for all elements in $C$. Therefore, all elements of $C$ are in $A\cap B$ and hence $C\SUB A\cap B$.
		\end{itemize}
	\end{proof}

	% 12
	\solution
	\begin{proof}
		Assume that $A\SUB B$ and $B\SUB C$, i.e. that $\forall x(x\in A\IMP x\in B)$ and $\forall x(x\in B\IMP x\in C)$. Take an arbitrary element $x$ in $A$. Because $A\SUB B$, $x$ is in $B$ as well. Moreover, because $B\SUB C$, we have that $x$ is in $C$ as well. Because $x$ is an arbitrary element of $A$, it holds for all elements of $A$ that they are in $C$ as well. Therefore, $A\SUB C$.
	\end{proof}

	% 13
	\solution
	\begin{proof}
		Let us assume that $A\SUB B$. Let us take an arbitrary element $a$ in $\POW(A)$. Since $a$ is in the power set of $A$, it is a subset of $A$. From $a\SUB A$, $A\SUB B$ and the previous question, we know that $a\SUB B$. Because of this, $a\in\POW(B)$. Because $a$ is an arbitrary element from $\POW(A)$, this is true for all elements in $\POW(A)$ and hence, $\POW(A)\SUB\POW(B)$.
	\end{proof}

	% 14
	\solution
	\begin{proof}
		We proof this by proof by contradiction. Therefore, assume that $P(M)$ and $P(k)\IMP P(k+1)$. We assume that $P(n)$ is not true for all $n\geq M$. Then there exists at least one number $s\geq n$ for which $P(s)$ is false. Let us take the smallest number $s^*\geq n$ such that $P(s^*)$ is false. Now, because $s^*$ is the smallest number, we have that $s^*-1$ is true. However we know that $P(k)\IMP P(k+1)$ holds for all $k\geq M$, thus also for $s^*-1$. Thus $P(s^*-1)\IMP P(s^*)$ holds, we now have that $P(s^*)$ is true, leading to a contradiction. Therefore, $P(n)$ must be true for all $n\geq M$.
	\end{proof}

	% 15
	\solution
	\begin{proof}
		Let $C(\phi)$ denote the number of connectives in $\phi$, and $V(\phi)$ denote the number of propositional variables. Let $P(\phi)$ be the statement that $V(\phi)\leq C(\phi)+1$, i.e. the number of propositional variables is at most one more than the number of connectives.\\
		\textbf{Base Case:} $V(x)=1$\\
		By the Atoms rule by the definition of $PROP$, $C(\phi)=0$. Therefore, the number of variables ($1$) is obviously at most one more than the number of connectives ($0$). Therefore, $P(\phi)$ holds.\\
		\textbf{Inductive Case:}\\
		Assume that we have two formulas $x,y\in PROP$, for which $P(x)$ and $P(y)$ are true, i.e. $V(x)\leq C(x)+1$ and $V(y)\leq C(y)+1$. We want to show that $P(\NOT x)$ and $P(x * y)$ for $*\in \{\IMP, \land, \lor\}$ holds as well. We will split the prove into a proof for the negation and the other connectives:
		\begin{itemize}
			\item $\NOT$:\\We want to show that $P(\NOT x)$ holds.
			\[V(\NOT x)=V(x)\overset{\text{IH}}\leq C(x)+1\leq C(x) + 2 = C(\NOT x) + 1\]. From this, we see that $V(\NOT x)\leq C(\NOT x) + 1$, hence $P(\NOT x)$ holds.
			\item $\IMP, \land, \lor$:\\
			In this case, we want to show that $P(x * y)$ holds for $*\in \{\IMP, \land, \lor\}$. \[V(x * y)=V(x) + V(y)\overset{\text{IH}}\leq C(x) + 1 + C(y) + 1=C(x)+C(y)+2=C(x*y) + 1\] This shows that $V(x*y)\leq C(x*y)+1$, and hence, $P(x*y)$ is true for $*\in \{\IMP, \land, \lor\}$.
		\end{itemize}
		
		Altogether, we have shown that $P(x)$ holds for an atom $x$ and that for all $x,y\in PROP$ and $*\in \{\IMP, \land, \lor, \NOT\}$, $P(x)\land P(y)\IMP P(\NOT x)\land P(x*y)$. Therefore, by the principle of structural induction, $P(\phi)$ is true for all $\phi\in PROP$, so for all propositional formula the number of propositional variables is at most one more than the number of connectives. This completes the proof by structural induction.
	\end{proof}
\end{solutions}

\section*{Solutions~4.2}
\begin{solutions}
	% 1
	\solution \begin{align*}
	x\in A\cup (B\cup C)&\IFF x\in A \OR (x\in B \OR x\in C) &&\text{(definition of $\cup$)}\\
	&\IFF (x\in A\OR x\in B)\OR x\in C &&\text{(associativity of $\OR$)}\\
	&\IFF x\in (A\cup B) \cup C &&\text{(definition of $\cup$)}
	\end{align*}
	\begin{align*}
	x\in A\cap (B\cap C)&\IFF x\in A \AND (x\in B \AND x\in C) &&\text{(definition of $\cap$)}\\
	&\IFF (x\in A\AND x\in B)\AND x\in C &&\text{(associativity of $\AND$)}\\
	&\IFF x\in (A\cap B) \cap C &&\text{(definition of $\cap$)}
	\end{align*}

	% 2
	\solution \begin{align*}
	x\in A&\IMP x\in A \OR x\in B\\
	&\IFF x\in A\cup B &&\text{(definition of $\cup$)} \\ \\
	x\in A\cap B&\IFF x\in A\AND x\in B && \text{(definition of $\cap$)}\\
	&\IMP x\in A \\
	\end{align*}

	% 3
	\solution \begin{align*}
	A\bigtriangleup B&\IFF \{x\st (x\in A)\XOR (x\in B)\} && \text{(definition of $\bigtriangleup$)}\\
	&\IFF \{x\st (x\in A \AND \NOT(x\in B)) \OR (\NOT(x\in A)\AND x\in B)\} &&\text{(definition of $\XOR$)}\\
	&\IFF \{x \st (x\in A \AND x\not\in B) \OR (x\not\in A\AND x\in B)\} &&\text{(definition of $\not\in$)}\\
	&\IFF \{x \st (x\in A\SETDIFF B) \OR (x\in B\SETDIFF A)\} &&\text{(definition of $\SETDIFF$)}\\
	&\IFF (A\SETDIFF B)\cup (B\SETDIFF A) &&\text{(definition of $\cup$)}
	\end{align*}
	
	% 4
	\solution \begin{tikzpicture}
	\newcommand{\mycolor}{red}
	\newcommand{\mytextcolor}{blue}
	\newcommand{\myformulacolor}{gray}
	
		\def\universe{(-2.0,-2) rectangle (2.0,2)};
		\def\setA{(-0.5,0) circle (1)}
		\def\setB{(0.5,0) circle (1)}
		\def\setC{(0,-0.8) circle (1)}

		\draw[step=1.0cm,gray] \universe;
		
		\begin{scope}
			\fill[\mycolor] \setA;
			\fill[\mycolor] \setB;
			\fill[\mycolor] \setC;
			\begin{scope}[even odd rule]
			\clip \setB;
			\fill[white] \setA;
			\end{scope}
			\begin{scope}[even odd rule]
			\clip \setC;
			\fill[white] \setA;
			\end{scope}
			\begin{scope}[even odd rule]
			\clip \setB;
			\fill[white] \setC;
			\end{scope}
			\begin{scope}[even odd rule]
			\clip \setC \setB \setA;
			\fill[\mycolor] \setA;
			\end{scope}
		\end{scope}
		\draw \setA node[above left] {};
		\draw \setB node[above right] {};
		\draw \setC node[below] {};
		\node at (150:1)	{\textbf{A}};
		\node at (30:1)	{\textbf{B}};
		\node at (-90:1.5)	{\textbf{C}};
	\end{tikzpicture}
	
	% 5
	\solution \begin{align*}
		\overline{\overline{A}} &\IFF \{x\in U\st x\not\in \overline{A}\} && \text{(definition of complement)}\\
		&\IFF \{x\in U\st \NOT(x\in \overline{A})\} && \text{(definition of $\not\in$)}\\
		&\IFF \{x\in U\st \NOT(x\not\in A) \} && \text{(definition of complement)}\\
		&\IFF \{x\in U\st \NOT\NOT(x\in A)\} && \text{(definition of $\not\in$)}\\
		&\IFF \{x\in U\st x\in A\} && \text{(definition of double negation)}\\
		&\IFF A\\ \\
		A\cup \overline{A} &\IFF \{x \in U\st x\in A \OR x\in \overline{A}\} &&\text{(definition of $\cup$)}\\
		&\IFF \{x\in U\st x\in A \OR x\not\in A\} && \text{(definition of complement)}\\
		&\IFF \{x\in U\st x\in A \OR \NOT(x\in A)\} && \text{(definition of $\not\in$)}\\
		&\IFF \{x\in U\st \T\} &&\text{(excluded middle ($p\OR \NOT p\equiv \T$))}\\
		&\IFF U 
	\end{align*}
	
	% 6
	\solution\begin{align*}
	\overline{A\cap B} &\IFF \{x\in U\st x\not\in A\cap B \} &&\text{(definition of complement)}\\
	&\IFF \{x\in U\st \NOT (x\in A\cap B)\} &&\text{(definition of $\not\in$)}\\
	&\IFF \{x\in U\st \NOT (x\in A\AND x\in B)\} &&\text{(definition of $\cap$)}\\
	&\IFF \{x\in U\st (\NOT (x\in A))\OR (\NOT (x\in B))\} &&\text{(DeMorgan's Law for logic)}\\
	&\IFF \{x\in U\st (x\not\in A)\OR (x\not\in B)\} &&\text{(definition of $\not\in$)}\\
	&\IFF \{x\in U\st (x\in\overline{A})\OR (x\in\overline{B})\}&&\text{(definition of complement)}\\
	&\IFF \overline{A}\cup\overline{B}&&\text{(definition of $\cup$)}
	\end{align*}

	% 7
	\solution \spart \begin{align*}
	(p\AND q)\IFF p&\equiv((p\AND q)\IMP p) \AND (p\IMP (p\AND q)) && \text{(definition of $\IFF$)}\\
	&\equiv (\NOT(p\AND q) \OR p)\AND (\NOT p \OR (p\AND q)) && \text{(definition of $\IMP$)}\\
	&\equiv (\NOT p \OR \NOT q \OR p)\AND ((\NOT p \OR p) \AND (\NOT p \OR q)) && \text{(DeMorgan's law and Distributive law)}\\
	&\equiv \T \AND (\NOT p\OR q) &&\text{($p\OR\NOT p\equiv \T$)}\\
	&\equiv p\IMP q &&\text{(definition of $\IMP$)}\\ \\
	(p\OR q)\IFF q&\equiv ((p\OR q)\IMP q) \AND (q\IMP (p\OR q)) && \text{(definition of $\IFF$)}\\
	&\equiv (\NOT (p\OR q) \OR q) \AND (\NOT q \OR (p\OR q)) &&\text{(definition of $\IMP$)}\\
	&\equiv (\NOT p\OR q)\AND \T &&\text{(DeMorgan's law and Distributive law)}\\
	&\equiv p\IMP q
	\end{align*}
	\spart In order to show that these three statements are equivalent, we show that $A\SUB~B \IMP A\cap B= A$, $A\cap B= A\IMP A\cup B= B$, and $A\cup B = B\IMP A\SUB B$:\begin{itemize}
		\item $A\SUB~B \IMP A\cap B= A$:\\
		We show this by contradiction, and therefore assume that $A\SUB B$ and that $A\cap B\neq A$. Because of the latter, we know that there is an element in $A$ which is not in $B$. However, this contradicts our assumption that $A\SUB B$, hence we know that the original implication is true.
		\item $A\cap B= A\IMP A\cup B= B$:\\
		We show this by contradiction, and therefore we assume that $A\cap B= A$ and $A\cup B\neq B$. From the latter, we know that there now exists an element in $A$ which is not in $B$, lets say $x$. However, this means that $x$ should also be excluded from $A\cap B$, and hence $A\cap B\neq A$, contradicting our assumption. Therefore, the original implication is true.
		\item $A\cup B = B\IMP A\SUB B$:\\
		We show this by contradiction, and therefore assume that $A\cup B = B$ and that $\NOT(A\SUB B)$. Because of the latter, there exists at least one element, let say $x$, such that $x\in A$ and $x\not\in B$. This means that $C=A\SETDIFF B\neq\emptyset$. However, this means that $A\cup B = B\cup C$. This contradicts our other assumption that $A\cup B = B$, which states that $A\cup B$ only contains elements of $B$, whereas we have derived that this is not possible. Because of this contradiction, we conclude that $A\cup B = B\IMP A\SUB B$.
	\end{itemize}
	Because we have shown that all three implications hold, we have now shown that the three statements are logically equivalent.
	
	
	% 8
	\solution\begin{align*}
	\overline{A}&\IFF \{x\in U\st x\not\in A\} &&\text{(definition of $\overline{A}$)}\\
	&\IFF \{x\st x\in U \AND x\not\in A\}\\
	&\IFF U\SETDIFF A&&\text{(definition of $\SETDIFF$)}\\ \\
	C\SETDIFF (A\cup B)&\IFF \{x\st x\in C \AND x\not\in (A\cup B)\} &&\text{(definition of $\SETDIFF$)}\\
	&\IFF \{x\st x\in C\AND\NOT (x\in (A\cup B))\}&&\text{(definition of $\not\in$)}\\
	&\IFF \{x\st x\in C\AND\NOT (x\in A\OR x\in B)\}&&\text{(definition of $\cup$)}\\
	&\IFF \{x\st x\in C \AND (x\not\in A \AND x\not\in B)\}&&\text{(DeMorgan's law)}\\
	&\IFF \{x\st (x\in C\AND x\not\in A)\AND (x\in C\AND x\not\in B)\}&&\text{(Distributive law)}\\
	&\IFF \{x\st (x\in C\SETDIFF A)\AND (x\in C\SETDIFF B)\}&&\text{(definition of $\SETDIFF$)}\\
	&\IFF \{x\st x\in (C\SETDIFF A)\cap (C\SETDIFF B)\}&&\text{(definition of $\cap$)}\\
	&\IFF (C\SETDIFF A)\cap (C\SETDIFF B)\\ \\
	C\SETDIFF (A\cap B)&\IFF \{x\st x\in C \AND x\not\in (A\cap B)\} &&\text{(definition of $\SETDIFF$)}\\
	&\IFF \{x\st x\in C\AND\NOT (x\in (A\cap B))\}&&\text{(definition of $\not\in$)}\\
	&\IFF \{x\st x\in C\AND\NOT (x\in A\AND x\in B)\}&&\text{(definition of $\cap$)}\\
	&\IFF \{x\st x\in C \AND (x\not\in A \OR x\not\in B)\}&&\text{(DeMorgan's law)}\\
	&\IFF \{x\st (x\in C\AND x\not\in A)\OR (x\in C\AND x\not\in B)\}&&\text{(Distributive law)}\\
	&\IFF \{x\st (x\in C\SETDIFF A)\OR (x\in C\SETDIFF B)\}&&\text{(definition of $\SETDIFF$)}\\
	&\IFF \{x\st x\in (C\SETDIFF A)\cup (C\SETDIFF B)\}&&\text{(definition of $\cup$)}\\
	&\IFF (C\SETDIFF A)\cup (C\SETDIFF B)
	\end{align*}

	% 9
	\solution\begin{align*}
	A\cup (A\cap B)&\IFF (A\cup A)\cap (A\cup B)&&\text{(Distributive law)}\\
	&\IFF A\cap (A\cup B)&&\text{(Idempotent law)}\\
	&\IFF A &&\text{($A\subseteq A\cup B$ (see 2))}
	\end{align*}

	% 10
	\solution\spart\begin{align*}
	X\cup (Y\cup X)&\IFF (X\cup (X \cup Y)) &&\text{(Commutative law)}\\
	&\IFF (X\cup X)\cup Y &&\text{(Associative law)}\\
	&\IFF X\cup Y &&\text{(Idempotent law)}
	\end{align*}
	\spart\begin{align*}
	(X\cap Y)\cap\overline{X}&\IFF (Y\cap X)\cap\overline{X}&&\text{(Commutative law)}\\
	&\IFF Y\cap (X\cap\overline{X})&&\text{(Associative law)}\\
	&\IFF Y\cap \emptyset&&\text{(Miscellaneous law $A\cap \overline{A}=\emptyset$)}\\
	&\IFF \emptyset &&\text{(Miscellaneous law $A\cap\emptyset=\emptyset$)}
	\end{align*}
	\spart\begin{align*}
	(X\cup Y)\cap \overline{Y}&\IFF(X\cap \overline{Y})\cup (Y\cap\overline{Y})&&\text{(Distribution law)}\\
	&\IFF (X\cap\overline{Y})\cup\emptyset&&\text{(Miscellaneous law $A\cap\overline{A}=\emptyset$)}\\
	&\IFF X\cap\overline{Y}&&\text{(Miscellaneous law $A\cup\emptyset = A$)}
	\end{align*}
	\spart\begin{align*}
	(X\cup Y)\cup (X\cap Y)&\IFF(X\cup (X\cap Y)\cup (Y\cup (X\cap Y))&&\text{(Distributive law)}\\
	&\IFF((X\cap X)\cup (X\cap Y))\cup((Y\cap X)\cup (Y\cap Y))&&\text{(Distributive law)}\\
	&\IFF (X\cup (X\cap Y))\cup ((Y\cap X)\cup Y)&&\text{(Idempotent law)}\\
	&\IFF X\cup Y&&\text{($A\subseteq A\cup B$ (see 2))}
	\end{align*}

	% 11
	\solution\spart\begin{align*}
	\overline{A\cup B\cup C}&\IFF \overline{A}\cap\overline{B}\cap\overline{C}&&\text{(Theorem 4.5)}
	\end{align*}
	\spart\begin{align*}
	\overline{A\cup B\cap C}&\IFF\overline{A}\cap\overline{B\cap C}&&\text{(DeMorgan's law)}\\
	&\IFF\overline{A}\cap(\overline{B}\cup\overline{C})&&\text{(DeMorgan's law)}
	\end{align*}
	\spart\begin{align*}
	\overline{\overline{A\cup B}}&\IFF\overline{\overline{A}\cap\overline{B}}&&\text{(DeMorgan's law)}\\
	&\IFF A\cup B&&\text{(DeMorgan's law)}
	\end{align*}
	\spart\begin{align*}
	\overline{B\cap\overline{C}}&\IFF\overline{B}\cup C&&\text{(DeMorgan's law)}
	\end{align*}
	\spart\begin{align*}
	\overline{A\cap\overline{B\cap\overline{C}}}&\IFF\overline{A\cap(\overline{B}\cup C)}&&\text{(DeMorgan's law)}\\
	&\IFF\overline{A}\cup \overline{(\overline{B}\cup C)}&&\text{(DeMorgan's law)}\\
	&\IFF\overline{A}\cup (B\cap\overline{C})
	\end{align*}
	\spart\begin{align*}
	A\cap\overline{A\cup B}&\IFF A\cap\overline{A}\cap\overline{B}&&\text{(DeMorgan's law)}\\
	&\IFF \emptyset\cap\overline{B}&&\text{(Miscellaneous law $A\cap\overline{A}=\emptyset$)}\\
	&\IFF\emptyset&&\text{(Miscellaneous law $A\cap\emptyset = \emptyset$)}
	\end{align*}

	% 12
	\solution\begin{proof}
		We give a proof by induction.  In the base case, $n=2$, the
		statement is that $\overline{X_1\cap X_2}=\overline{X_1}\cup\overline{X_n}$.
		This is true since it is just an application of DeMorgan's law for two sets.

		For the inductive case, suppose that the statement is true for $n=k$. Hence, we assume the induction hypothesis: $\overline{X_1\cap X_2\cap \cdots \cap X_{k}}=\overline{X_1}\cup\overline{X_2}\cup\cdots\cup\overline{X_{k}}$,
		for $X_1$, $X_2$,
		\dots, $X_{k+1}$ being any $k+1$ sets. Then we have:
		\begin{align*}
		\overline{X_1\cap X_2\cap \cdots \cap X_{k+1}}
		&= \overline{(X_1\cap X_2\cap \cdots \cap X_k) \cap X_{k+1}}\\
		&= \overline{(X_1\cap X_2\cap \cdots \cap X_k)}\cup\overline{X_{k+1}}\\
		&= (\overline{X_1}\cup\overline{X_2}\cup\cdots\cup\overline{X_k})\cup\overline{X_{k+1}}&&\text{(IH)}\\
		&= \overline{X_1}\cup\overline{X_2}\cup\cdots\cup\overline{X_{k+1}}
		\end{align*}
		In this computation, the second step follows by DeMorgan's Law for
		two sets, while the third step follows from the induction hypothesis. Therefore by the principle of induction we have
		proven the theorem.
	\end{proof}

	% 13
	\solution\begin{itemize}
		\item For any natural number $n\geq 2$ and any sets $Q, P_1, P_2,\ldots, P_n$: $Q\cap (P_1\cup P_2\cup \ldots \cup P_n)=(Q\cap P_1)\cup (Q\cap P_2)\cup\ldots\cup (Q\cap P_n)$\begin{proof}
			We proof this by using induction. In the base case, $n=2$, the statement is that $Q\cap (P_1\cup P_2)=(Q\cap P_1)\cup (Q\cap P_2)$. This is true since this is just an application of the Distributive law for three sets.

			For the inductive case, suppose that the statement is true for $n=k$, where $k$ is an arbitrary integer bigger or equal to $2$. Hence, we assume the induction hypothesis: $Q\cap (P_1\cup P_2\cup \ldots \cup P_k)=(Q\cap P_1)\cup (Q\cap P_2)\cup\ldots\cup (Q\cap P_k)$, for $Q$, $P_1, P_2,\ldots P_{k+1}$ being any $k+2$ sets. Then we have:
			\begin{align*}
				Q\cap (P_1\cup P_2\cup \ldots \cup P_{k+1})&=(Q\cap (P_1\cup P_2\cup \ldots\cup P_k))\cup (Q\cap P_{k+1})\\
				&=((Q\cap P_1)\cup (Q\cap P_2)\cup\ldots\cup (Q\cap P_k))\cup (Q\cap P_{k+1})&&\text{(IH)}\\
				&=(Q\cap P_1)\cup (Q\cap P_2)\cup\ldots\cup (Q\cap P_{k+1})
			\end{align*}

			In this computation, the second step follows by Distributive law for three sets, while the third step follows from the induction hypothesis. Therefore by the principle of induction we have proven the theorem.
		\end{proof}
		\item For any natural number $n\geq 2$ and any sets $Q, P_1, P_2,\ldots, P_n$: $Q\cup (P_1\cap P_2\cap \ldots \cap P_n)=(Q\cup P_1)\cap (Q\cup P_2)\cap\ldots\cap (Q\cup P_n)$\begin{proof}
		We proof this by using induction. In the base case, $n=2$, the statement is that $Q\cup (P_1\cap P_2)=(Q\cup P_1)\cap (Q\cup P_2)$. This is true since this is just an application of the Distributive law for three sets.

		For the inductive case, suppose that the statement is true for $n=k$, where $k$ is an arbitrary integer bigger or equal to $2$. $Q\cup (P_1\cap P_2\cap \ldots \cap P_k)=(Q\cup P_1)\cap (Q\cup P_2)\cap\ldots\cap (Q\cup P_k)$, for $Q$, $P_1, P_2,\ldots P_{k+1}$ being any $k+2$ sets. Then we have:
		\begin{align*}
		Q\cup (P_1\cap P_2\cap \ldots \cap P_{k+1})&=(Q\cup (P_1\cap P_2\cap \ldots\cap P_k))\cap (Q\cup P_{k+1})\\
		&=((Q\cup P_1)\cap (Q\cup P_2)\cap\ldots\cap (Q\cup P_k))\cap (Q\cup P_{k+1})&&\text{(IH)}\\
		&=(Q\cup P_1)\cap (Q\cup P_2)\cap\ldots\cap (Q\cup P_{k+1})
		\end{align*}

		In this computation, the second step follows by Distributive law for three sets, while the third step follows from the induction hypothesis. Therefore by the principle of induction we have proven the theorem.
	\end{proof}
	\end{itemize}
\end{solutions}

\section*{Solutions 4.4}
\begin{solutions}
	% 1
	\solution\begin{align*}
	A\times B&=\{(1,a), (1,b), (1,c), (2,a), (2,b), (2,c), (3,a),(3,b),(3,c),(4,a),(4,b),(4,c)\}\\
	B\times A&=\{(a,1),(a,2),(a,3),(a,4),(b,1),(b,2),(b,3),(b,4),(c,1),(c,2),(c,3),(c,4)\}
	\end{align*}

	% 2
	\solution\begin{align*}
	g\circ f=\{(a,c), (b,c),(c,b),(d,d)\}
	\end{align*}

	% 3
	\solution\begin{align*}
	B^A=&\{\{(a,0), (b,0), (c,0)\},
	\{(a,0), (b,0), (c,1)\},
	\{(a,0), (b,1), (c,0)\},
	\{(a,0), (b,1), (c,1)\},\\
	&\{(a,1), (b,0), (c,0)\},
	\{(a,1), (b,0), (c,1)\},
	\{(a,1), (b,1), (c,0)\},
	\{(a,1), (b,1), (c,1)\}\}
	\end{align*}

	% 4
	\solution
		\spart $f$ is not onto, as there exists no element $x$ in $\Z$ such that $f(x)=2x=3$, because this means that $x=1.5$, which is not an integer. However, it is one-to-one. Take two arbitrary $a$ and $b$ such that $f(a)=f(b)$. Hence, $2a=2b$, which can only be true if $a=b$.
		\spart $g$ is onto; take an arbitrary $y$ in $\Z$. Then there exists an $x$ for which $g(x)=y$, namely $x=y-1$ ($g(x)=g(y-1)=y-1+1=y$), which is integer and thus in $ \Z$. Moreover, $g$ is one-to-one as well. Take two arbitrary $a$ and $b$ such that $g(a)=g(b)$. Hence $a+1=b+1$, which can only be true when $a=b$.
		\spart $h$ is not onto, as there exists no element $x$ in $\Z$ such that $h(x)=x^2+x+1=4$. This is because $x^2+x+1=4\IFF x=\frac{\pm\sqrt{13}-1}{2}$, which is not an integer. It is not one-to-one either, as solving $x^2+x+1=a$ gives two solutions for every $a$. Let us take $a=3$, then both $x=-2$ and $y=1$ give $h(x)=h(y)=3$, however, $x\neq y$.
		\spart $s$ is onto; take an arbitrary $y\in\Z$. Then there exists an $x$ for which $s(x)=y$, namely $x=2y$ ($s(x)=s(2y)=\frac{2y}{2}=y$). However, it is not one-to-one. Take an arbitrary even integer $a$ and $b=a-1$. $s(a)=\frac{a}{2}=\frac{(a-1)+1}{2}=s(b)$, however, $a\neq b$.

	% 5
	\solution For any $x\in A$:
	\begin{align*}
	((h\circ g)\circ f)(x)=(h\circ g)(f(x))=(h(g(f(x))))=h((g\circ f)(x))=(h\circ (g\circ f))(x)
	\end{align*}

	% 6
	\solution\spart\textbf{To prove:} $g\circ f$ is one-to-one $\IMP f$ is one-to-one
	\begin{proof}
		We use proof by contradiction. We assume that $g\circ f$ is one-to-one, but $f$ is not. Because $f$ is not one-to-one, so there exists an $a,b\in A$ such that $f(a)=y=f(b)$, but $a\neq b$. However, since $g: B\to C$, we have an element $x\in C$ such that $g(y)=x$. However, this would mean that for both $a$ and $b$, $(g\circ f)(a)=(g\circ f)(b)=x$, showing that $g\circ f$ is not one-to-one. However, this contradicts our assumption that $g\circ f$ is one-to-one, which mean that it cannot hold that $g\circ f$ is one-to-one, but $f$ is not. Hence, the statement that has to be proven is true.
	\end{proof}
	\spart Let $A=\{1\}, B=\{a, b\}, C=\{c\}, f(1)=a, g(a)=g(b)=c$. Although $g$ is not one-to-one, as both $g(a)=g(b)=c$, we have that $g\circ f(1)=c$, which is one-to-one.

	% 7
	\solution\spart \textbf{To prove:} $g\circ f$ is onto $\IMP g$ is onto\begin{proof}
		We use proof by contradiction. We assume that $g\circ f$ is onto, but $g$ is not. Because $g$ is not onto, this means that there is an element $c\in C$ such that for all elements in $b\in B: g(b)\neq c$. However, this would mean that $g\circ f$ cannot be onto, as there is no element in $B$ that $f$ can map to such that $g\circ f(x)=c$. However, this contradicts our assumption that $g\circ f$ is onto, which mean that it cannot hold that $g\circ f$ is onto, but $g$ is not. Hence, the statement that has to be proven is true.
	\end{proof}
	\spart Let $A=\{a\}, B=\{b,c\}, C=\{d\}, f(a)=b, g(b)=d$. Although $f$ is not onto, as there is no $x\in A$ such that $f(x)=c$, we have that for all elements in $C$, namely $d$, that there exists an element in $A$, namely $a$, such that $(g\circ f)(a)=d$. Hence $g\circ f$ is onto.
\end{solutions}

\tipbox{Couldn't find your answer here? Feel free to submit your own for future editions of the book here:
\url{https://gitlab.ewi.tudelft.nl/reasoning_and_logic/book_solutions}. We will add your name to the list of
contributors for the book if we accept your answers!}


\subsection*{Contributors to Solutions}

\emph{Max van Deursen},\\
\emph{Kevin Chong},\\
\emph{Julian Kuipers},\\
\emph{Pia Keukeleire},\\
\emph{Philippos Boon Alexaki},\\
\emph{Thijs Schipper}

\end{document}
