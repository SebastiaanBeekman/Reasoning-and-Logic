\section*{Solutions~3.7}%\ref{S-logic-4}}

\begin{solutions}
	
	\setcounter{solutioncounter}{0}
	\solution 
	
	\textit{Proof:} We prove this once again by induction. 
	
	\begin{itemize}
		\item Base Case: When $n = 0$ and when $n = 1$, the statement is clearly true, since $f_0 = 0 < 1 = 2^0$ and $f_1 = 1 < 2 = 2^1$, so $f_0 < 2^0$ and $f_1 < 2^1$.
		
		\item Inductive Case: If we now take an arbitrary $k$ such that $k \geq 2$, we assume that $f_n < 2^n$ for $n = k-1$ and $n = k-2$ holds. 
		To complete the induction, we need to show that $f_k < 2^k$ is also true: 
	\end{itemize}
	\begin{align*}
		f_k &= f_{k-1} + f_{k-2} \\
		&< 2^{k-1} + 2^{k-2}  &\textnormal{(inductive hypothesis)} \\
		&= \frac{2^{k}}{2} + \frac{2^{k}}{4} \\
		&= \frac{3}{4} \cdot 2^k \\
		&< 2^k
	\end{align*}	
	
		This shows that $f_k < 2^k$ is true, which completes the induction. 
		
	\solution 
	
	\textit{Proof:} We prove the equation using induction. 
	
	\begin{itemize}
		\item Base Case: When $n = 1$, $a_1 = 1\cdot2^{1-1}$ is true since both sides are equal to 1. 
		
		\item Inductive Case: Let $k \geq 1$ be an arbitrary integer. We asssume that $a_n = n2^{n-1}$ holds for $n = k-1$. 
		To complete the induction, we need to show that the equation also holds for $n = k$.
	\end{itemize}
	\begin{align*}
		a_k &= 2a_{k-1} + 2^{k-1} \\
		&= 2(k-1)2^{k-1-1} + 2^{k-1} &\text{(inductive hypothesis)} \\
		&= (k-1)2^{k-1} + 2^{k-1} \\
		&= ((k-1)+1)2^{k-1} \\
		&= k2^{k-1}
	\end{align*}	
	
	Which proves that the equation also holds for $n=k$. Thereby the induction is completed.  
	
\end{solutions}

