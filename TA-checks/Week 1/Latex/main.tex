\documentclass[a4paper]{article}

\usepackage{graphicx} 
\usepackage[english]{babel}
\usepackage{graphicx}
\usepackage{float}
\usepackage{amssymb}
\usepackage[a4paper,top=3cm,bottom=2cm,left=2cm,right=2cm,marginparwidth=1.75cm]{geometry}

\begin{document}

\begin{titlepage}
    \newcommand{\HRule}{\rule{\linewidth}{0.5mm}}
    \center

    \textsc{\LARGE Delft University of Technology}\\[1cm]

    \textsc{\Large Reasoning \& Logic}\\[0.2cm]
    \textsc{\large CSE1300}\\[1cm]
    \HRule \\[0.8cm]
    { \huge \bfseries Assignment: TA-check 1}\\[0.7cm]
    \HRule \\[2cm]
    \large
    \emph{Authors:}\\
    Joris Rijs (5880998) \& Sebastiaan Beekman (5885116)\\[1.5cm]
    {\large \today}\\[5cm]
    \includegraphics[width=0.6\textwidth]{images/TU_delft_logo.jpg}\\[1cm]
    \vfill
\end{titlepage}

\newpage
\tableofcontents

\newpage
\section{Question 1}
Consider the following claims and arguments written in English. Formulate a logically precise claim in
propositional logic using logical operators to represent the claim. For instance for “I like puzzles and I
like tea”, one might use p $\wedge $ q where p is “I like puzzles” and r is “I like tea.”.
\\\\
\textbf{(a). I do this homework, then I have a greater chance at passing the course}
\begin{itemize}
    \item p = "I do this homework"
    \item q = "I have a greater chance at passing the course"
    \item p $\rightarrow $ q
\end{itemize}
\ \\
\textbf{(b). Only if I do this homework will I get feedback from TA's}
\begin{itemize}
    \item p = "I do this homework"
    \item q = "I get feedback from TA's"
    \item p $\leftrightarrow $ q
\end{itemize}
\ \\
\textbf{(c). I can choose to do the MC-test in week 3 or I can choose not to}
\begin{itemize}
    \item p = "I can choose to do the MC-test in week 3"
    \item p $\vee $ $\neg $p
\end{itemize}
\ \\
\textbf{(d).}\\
If I pass the course, then I have practiced well.\\
If I pass the course, then I have passed the endterm.\\
\underline{I have practiced well and passed the endterm.}\\
Therefore, I have passed the course.
\begin{itemize}
    \item p = "I pass the course"
    \item q = "I practice well"
    \item r = "I pass the endterm"
    \item ((p $\rightarrow $ q) $\wedge $ (p $\rightarrow $ r) $\wedge $ (q $\wedge $ r)) $\rightarrow $ p
\end{itemize}
\ \\
\textbf{(e).}\\
It is not true that: I do the homework and I do not get feedback from TA's.\\
I do the homework or I do not have a greater chance of passing the course.\\
\underline{I do not get feedback from the TA's}\\
Therefore, I do not have a greater chance of passing the course.
\begin{itemize}
    \item p = "I do the homework"
    \item q = "I get feedback from TA's"
    \item r = "I have a greater changce of passing the course"
    \item (($\neg $(p $\wedge $ ($\neg $q))) $\wedge $ (p $\vee $ ($\neg $r)) $\wedge $ ($\neg $q)) $\rightarrow $ ($\neg $r)
\end{itemize}

\newpage
\section{Question 2}
Answer each of the following questions in at most 5 lines of text (excluding truth tables if required). For
each of the questions (which consist of just an argument) indicate if the argument is valid or not. If they
are valid, prove it using a truth table. If they are invalid, prove it using a counterexample and explain
how this counterexample shows the argument to be invalid.
\\\\
\textbf{(a). Explain why we need a full truth table to show an argument is valid, but why only a
    counterexample suffices to show an argument is invalid}\\
An argument is valid if, and only if, in every situation where all premises are true, the conclusion is also true. This means that if we can find a situation where all premises are true, but the conclusion is false, the argument is invalid. This is why a counterexample suffices to show an argument is invalid.
\\\\
\textbf{(b). p $\rightarrow $ q, $\neg $q $\therefore $ $\neg $p (Valid)}\\
\begin{displaymath}
    \begin{array}{|c c|c|}
        p & q & ((p \rightarrow q) \wedge (\neg q)) \rightarrow (\neg p) \\
        \hline
        0 & 0 & 1                                                        \\
        0 & 1 & 1                                                        \\
        1 & 0 & 1                                                        \\
        1 & 1 & 1                                                        \\
    \end{array}
\end{displaymath}
\ \\
\textbf{(c). p $\rightarrow $ q, p $\rightarrow $ r, q $\rightarrow $ r, $\therefore $ p (Invalid)}\\
\begin{displaymath}
    \begin{array}{|c c c|c|}
        p & q & r & ((p \rightarrow q) \wedge (p \rightarrow r) \wedge (q \rightarrow r)) \rightarrow p \\
        \hline
        0 & 0 & 0 & 0                                                                                   \\
        0 & 0 & 1 & 0                                                                                   \\
        0 & 1 & 0 & 1                                                                                   \\
        0 & 1 & 1 & 0                                                                                   \\
        \hline
        1 & 0 & 0 & 1                                                                                   \\
        1 & 0 & 1 & 1                                                                                   \\
        1 & 1 & 0 & 1                                                                                   \\
        1 & 1 & 1 & 1                                                                                   \\
    \end{array}
\end{displaymath}
\ \\
\textbf{(d). $\neg $(p $\wedge $ ($\neg $q)), p $\vee $ ($\neg $r), $\neg $q $\therefore $ $\neg $r (Invalid)}\\
\begin{displaymath}
    \begin{array}{|c c c|c|}
        p & q & r & ((\neg (p \wedge (\neg q))) \wedge (p \vee (\neg r)) \wedge (\neg q)) \rightarrow (\neg r) \\
        \hline
        0 & 0 & 0 & 0                                                                                          \\
        0 & 0 & 1 & 1                                                                                          \\
        0 & 1 & 0 & 1                                                                                          \\
        0 & 1 & 1 & 1                                                                                          \\
        \hline
        1 & 0 & 0 & 1                                                                                          \\
        1 & 0 & 1 & 1                                                                                          \\
        1 & 1 & 0 & 1                                                                                          \\
        1 & 1 & 1 & 1                                                                                          \\
    \end{array}
\end{displaymath}
\ \\
\textbf{(e). Consider an argument without any premises. Explain when such an argument is valid and when such an argument is invalid}\\
An argument without premises is valid when the conclusion is a tautology. Otherswise, it is invalid.
\\\\
\textbf{(f). $\therefore $ (p $\rightarrow $ q) $\rightarrow $ $\neg $(p $\wedge $ $\neg $q) (Valid)}\\
\begin{displaymath}
    \begin{array}{|c c|c|}
        p & q & (p \rightarrow q) \rightarrow \neg (p \wedge \neg q) \\
        \hline
        0 & 1 & 1                                                    \\
        0 & 1 & 1                                                    \\
        1 & 0 & 1                                                    \\
        1 & 1 & 1                                                    \\
    \end{array}
\end{displaymath}
\ \\\\\\
\textbf{(g). Consider a new ternary operator p $\rightarrow q$ r, such that when q is false p $\rightarrow q$ r is equivalent to p $\leftrightarrow $ r, and when q is true p $\rightarrow q$ r is equivalent to p $\wedge $ $\neg $r. Give a truth table for p $\rightarrow q$ r (Invalid)}
\begin{displaymath}
    \begin{array}{|c c c|c|c|c|}
        p & q & r & p \leftrightarrow r & p \wedge \neg r & p \rightarrow q r \\
        \hline
        0 & 0 & 0 & 1                   & X               & 1                 \\
        0 & 0 & 1 & 0                   & X               & 0                 \\
        0 & 1 & 0 & X                   & 0               & 0                 \\
        0 & 1 & 1 & X                   & 0               & 0                 \\
        \hline
        1 & 0 & 0 & 0                   & X               & 0                 \\
        1 & 0 & 1 & 1                   & X               & 1                 \\
        1 & 1 & 0 & X                   & 1               & 1                 \\
        1 & 1 & 1 & X                   & 0               & 0                 \\
    \end{array}
\end{displaymath}
\newpage
\section{Question 3}
For each of the following pairs of statements, indicate if they are equivalent and explain why (not) using a truth table. In case of unequivalent statements an incomplete truth table suffices.
\\\\
\textbf{(a). (p $\vee $ (q  $\wedge $ r)) $\equiv ?$ ($\neg $p $\rightarrow $ (q $\wedge $ r)) (Valid)}\\
($\neg $p $\rightarrow $ (q $\wedge $ r)) $\equiv $ (p $\vee $ (q  $\wedge $ r)) | De Morgan's
\\\\
\textbf{(b). (p $\vee $ q) $\rightarrow $ (r $\vee $ q) $\equiv ?$ ($\neg $r $\vee $ $\neg $q) $\rightarrow $ ($\neg $p $\vee $ $\neg $q) (Invalid)}\\
\begin{displaymath}
    \begin{array}{|c c c|c|c|}
        p & q & r & (p \vee q) \rightarrow (r \vee q) & (\neg r \wedge \neg q) \rightarrow (\neg p \vee \neg q) \\
        \hline
        0 & 0 & 0 & 1                                 & 1                                                       \\
        0 & 0 & 1 & 1                                 & 1                                                       \\
        0 & 1 & 0 & 1                                 & 1                                                       \\
        0 & 1 & 1 & 1                                 & 1                                                       \\
        \hline
        1 & 0 & 0 & 0                                 & 1                                                       \\
        1 & 0 & 1 & 1                                 & 1                                                       \\
        1 & 1 & 0 & 1                                 & 0                                                       \\
        1 & 1 & 1 & 1                                 & 1                                                       \\
    \end{array}
\end{displaymath}
\ \\
\textbf{(c). (q $\vee $ (p  $\rightarrow $ q)) $\equiv ?$ $\neg $($\neg $q $\wedge $ (p $\wedge $ $\neg $q)) (Valid)}\\
(q $\vee $ $\neg $(p $\wedge $ $\neg $q)) $\equiv $ $\neg $($\neg $q $\wedge $ (p $\wedge $ $\neg $q))\\
(q $\vee $ ($\neg $p $\vee $ q))\\
(q $\vee $ (p $\rightarrow $ q))
\\\\
\textbf{(d). (($\neg $s $\leftrightarrow $ p) $\wedge $ (r $\rightarrow $ p)) $\equiv ?$ (($\neg $p $\wedge $ $\neg $r $\wedge $ $\neg $s) $\vee $ (p $\wedge $ s)) (Invalid)}\\
\begin{displaymath}
    \begin{array}{|c c c|c|c|}
        p & r & s & ((\neg s \leftrightarrow p) \wedge (r \rightarrow p)) & ((\neg p \wedge \neg r \wedge \neg s) \vee (p \wedge s)) \\
        \hline
        0 & 0 & 0 & 0                                                     & 1                                                        \\
        0 & 0 & 1 & 1                                                     & 0                                                        \\
        0 & 1 & 0 & 0                                                     & 0                                                        \\
        0 & 1 & 1 & 0                                                     & 0                                                        \\
        \hline
        1 & 0 & 0 & 1                                                     & 0                                                        \\
        1 & 0 & 1 & 0                                                     & 1                                                        \\
        1 & 1 & 0 & 1                                                     & 0                                                        \\
        1 & 1 & 1 & 0                                                     & 1                                                        \\
    \end{array}
\end{displaymath}


\newpage
\section{Question 4}
For every formula in propositional calculus it is possible to rewrite it to a formula in disjunctive normal form (DNF). A formula in this form consists of a disjunction of conjunctions, with the conjunctions consisting of loose atoms and/or their negations. Formula F = (p1 $\wedge $ $\neg $p2) contains a single conjunction and is therefore in disjunctive normal form.
\\\\
\textbf{(a). p $\vee $ $\neg $(q $\wedge $ r)}\\
p $\vee $ $\neg $q $\vee $ $\neg $r
\\\\
\textbf{(b). $\neg $(p $\wedge $ (q $\vee $ r))}\\
$\neg $p $\vee $ $\neg $(q $\vee $ r)\\
$\neg $p $\vee $ ($\neg $q $\wedge $ $\neg $r)
\\\\
\textbf{(c). $\neg $(p $\rightarrow $ q) $\wedge $ (p $\rightarrow $ r)}\\
$\neg $($\neg $p $\vee $ q) $\wedge $ ($\neg $p $\vee $ r)\\
(p $\wedge$ $\neg $q) $\wedge $ ($\neg $p $\vee $ r)\\
(p $\wedge $ $\neg $q $\wedge $ $\neg $p) $\vee $ (p $\wedge $ $\neg $q $\wedge $ r)\\
F $\vee $ (p $\wedge $ $\neg $q $\wedge $ r)\\
(p $\wedge $ $\neg $q $\wedge $ r)
\\\\
\textbf{(d). $\neg $((p $\wedge $ q) $\leftrightarrow $ (r $\vee $ s))}\\
$\neg $(((p $\wedge $ q) $\rightarrow $ (r $\vee $ s)) $\wedge $ ((r $\vee $ s) $\rightarrow $ (p $\wedge $ q)))\\
$\neg $((($\neg $p $\vee $ $\neg $q) $\vee $ (r $\vee $ s)) $\wedge $ (($\neg $r $\wedge $ $\neg $s) $\vee $ (p $\wedge $ q)))\\
$\neg $(($\neg $p $\vee $ $\neg $q) $\vee $ (r $\vee $ s)) $\vee $ $\neg $(($\neg $r $\wedge $ $\neg $s) $\vee $ (p $\wedge $ q))\\
($\neg $($\neg $p $\vee $ $\neg $q) $\wedge $ $\neg $(r $\vee $ s)) $\vee $ ($\neg $($\neg $r $\wedge $ $\neg $s) $\wedge $ $\neg $(p $\wedge $ q))\\
((p $\wedge $ q) $\wedge $ ($\neg $r $\wedge $ $\neg $s)) $\vee $ ((r $\vee $ s) $\wedge $ ($\neg $p $\vee $ $\neg $q))\\
(p $\wedge $ q $\wedge $ $\neg $r $\wedge $ $\neg $s) $\vee $ ((r $\vee $ s) $\wedge $ ($\neg $p $\vee $ $\neg $q))\\
Stuck...
\\\\

\newpage
\section{Question 5}
Translate the following statements to a first-order language. Each time, you should first name the
symbols you will use, along with their meaning, such as predicate symbols (usually capital letters
such as B(x,y) or Brother(x,y) for 'x is the brother of y,' or W(x,y) for 'x wants y'), and constants
(small letters without argument, such as j for 'John'). You can reuse symbols in later subquestions,
as long as you do not define new symbols with the same name but a different meaning.
You should think about what objects should be constants, and what objects should not be. For
example, if you were to translate: “Stefan is a lecturer”, then “Stefan” refers to a specific object in
your domain and should thus be a constant. On the other hand, being a lecturer is a property that
objects can (not) have and thus should be a predicate. Thus you could translate this as Lecturer psq
where s is Stefan and Lecturer pxq means “x is a lecturer”.
Remember that predicates should define only a single property, and not secretly introduce new
objects. E.g., A(x) for 'x owns an aviation license' is incorrect! This predicate now does two things,
both positing that an object exists which is an aviation license, and also showing a relation between
x and this license.
\\\\
\textbf{(I). All lawyers are happy}
\begin{itemize}
    \item H(x): "x is happy"
    \item Domain: [Lawyers]
    \item $\forall $x H(x)
\end{itemize}
\ \\
\textbf{(II). Phoenix is a lawyer}
\begin{itemize}
    \item L(x): "x is a lawyer"
    \item L(Phoenix)
\end{itemize}
\ \\
\textbf{(III). Maya eats a burger}
\begin{itemize}
    \item E(x,y): "x eats y"
    \item E(Maya, burger)
\end{itemize}
\ \\
\textbf{(IV). Maya eats burgers}
\begin{itemize}
    \item E(x,y): "x eats y"
    \item Domain: [Burgers]
    \item $\forall $x E(Maya, x)
\end{itemize}
\ \\
\textbf{(V). Maya was helped by Phoenix, who also helped Maya's cousin}
\begin{itemize}
    \item H(x,y): "x helped y"
    \item H(Phoenix, Maya) $\wedge $ H(Phoenix, Maya's cousin)
\end{itemize}
\ \\
\textbf{(VI). CSE1300 students are the only ones who know about Phoenix}
\begin{itemize}
    \item K(x,y): "x knows y"
    \item Domain: [CSE1300 Students]
    \item $\forall $x K(x, Phoenix)
\end{itemize}
\ \\
\textbf{(VII). Phoenix is never someone's only help}
\begin{itemize}
    \item H(x,y): "x helped y"
    \item $\exists $x $\exists $y (H(Phoenix, x) $\wedge $ H(y, x) $\wedge $ (Phoenix $\neq $ y))
\end{itemize}
\newpage
\textbf{(VIII). There is no lawyer without a badge who only defended one case}
\begin{itemize}
    \item L(x): "x is a lawyer"
    \item B(x): "x has a badge"
    \item D(x,y): "x defended y"
    \item $\exists $x $\exists $y $\forall $z (L(x) $\wedge $ $\neg $B(x) $\wedge $ D(x,y) $\wedge $ (D(x,z) $\rightarrow $ z=y))
\end{itemize}
\ \\
Suppose we have the domain D = \{a1, a2, a3, p1, p2, p3, p4, p5, p6, u1, u2, u3, u4\}, the predicates
D(x,y), P(x,y,z), U(x,y) and A(x,y). The interpretations of these symbols is as follows: D(x,y)
means paper x is stored in database y, P(x,y,z) means x has published paper y at university z,
U(x,y) means university x uses database y and lastly A(x,y) means author x wrote paper y.
Now form natural language sentences for the following first-order sentences. You are allowed to use
the constants from the domain in your sentences, or you can assign (consistent) natural names to
them.
\\\\
\textbf{(I). P(a1, p1, u3)}\\
a1 has published p1 at u3.
\\\\
\textbf{(II). $\exists $x P(a2, x, u2)}\\
a2 has published at least one paper at u2.
\\\\
\textbf{(III). $\exists $x$\forall $y$\forall $z ($\neg $(P(y, z, x)))}\\
There is a university that has no papers published by any authors.
\\\\
\textbf{(IV). $\forall $w$\forall $x$\forall $y$\forall $z ((P(w,x,y) $\wedge $ D(x,z)) $\rightarrow $ U(y,z)) }\\
If every paper published at an university can be found in a database, then that university uses said database.
\\\\
\textbf{(V). $\forall $x$\exists $y$\exists $z(A(y,x) $\wedge $ A(z, x) $\wedge $ $\neg $(y = z))}\\
Every paper has at least two authors.
\\\\

\newpage
\section{Question 6}
\textbf{(a). In your own words, describe the principle of explosion. Also provide an example of an argument that uses it}
The principle of explosion the principle that any statement can be proven to be true from a contradiction. For example:\\
P: "Grass is green" \& Q: "Stefan is a lizzard in a human suit"\\
P, $\neg $P, P $\vee $ Q $\therefore$ Q\\
We can prove that Q is true from the contradiction that P and $\neg $P are both true.\\
WE KNEW IT!
\\\\
\textbf{(b). Prove that with just \{$\neg$, $\rightarrow $\} we can emulate the \{$\wedge$, $\vee $, $\leftrightarrow $\} operators. In other words, show that \{$\neg$, $\rightarrow $\} is functionally complete}\\
P $\rightarrow $ Q $\equiv $ $\neg $P $\vee $ Q\\
($\neg$P) $\rightarrow $ Q $\equiv $ P $\vee $ Q\\
$\neg$(P $\rightarrow $ ($\neg$Q)) $\equiv $ P $\wedge $ Q\\
P $\leftrightarrow $ Q $\equiv $ (P $\rightarrow $ Q) $\wedge $ (Q $\rightarrow $ P) $\equiv $ $\neg$((p $\rightarrow $ Q) $\rightarrow $ $\neg$(Q $\rightarrow $ P))
\\\\
\textbf{(c). The operator p $\rightarrow q$ r from earlier in this homework is functionally complete. Prove this.}\\
Tried many different ways to prove this, but we can't seem to get it.
\\\\
\textbf{(d). Consider the truth table for a compound proposition with 7 unique atoms and 143 connectives. How many different configurations can the column for the main connective have?}\\
\(2^7\) = 128 rows.\\
\(2^128\) = 3.4 * \(10^38\) possible configurations.
\\\\
\textbf{(e). Does the order of quantifiers matter? Give an example to support your answer.}\\
Yes, the order of quantifiers matter. For example, $\forall $x $\exists $y P(x,y) is not the same as $\exists $y $\forall $x P(x,y). The first one means that for every x, there exists a y such that P(x,y), while the second one means that there exists a y such that for every x, P(x,y).
\\\\
\textbf{(f). Reflection}\\
Most diffecult: 4c, 4d (Mostly a lot of work which induces mistakes) \& 6c\\
Improve speed: 5
\end{document}