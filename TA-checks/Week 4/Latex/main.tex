\documentclass[a4paper]{article}

\usepackage{graphicx} 
\usepackage[english]{babel}
\usepackage{graphicx}
\usepackage{float}
\usepackage{logicproof}
\usepackage{amssymb}
\usepackage[a4paper,top=3cm,bottom=2cm,left=2cm,right=2cm,marginparwidth=1.75cm]{geometry}

\begin{document}

\begin{titlepage}
    \newcommand{\HRule}{\rule{\linewidth}{0.5mm}}
    \center

    \textsc{\LARGE Delft University of Technology}\\[1cm]

    \textsc{\Large Reasoning \& Logic}\\[0.2cm]
    \textsc{\large CSE1300}\\[1cm]
    \HRule \\[0.8cm]
    { \huge \bfseries Assignment: TA-check 2}\\[0.7cm]
    \HRule \\[2cm]
    \large
    \emph{Authors:}\\
    Joris Rijs (5880998) \& Sebastiaan Beekman (5885116)\\[1.5cm]
    {\large \today}\\[5cm]
    \includegraphics[width=0.6\textwidth]{images/TU_delft_logo.jpg}\\[1cm]
    \vfill
\end{titlepage}

\newpage
\tableofcontents

\newpage
\section{Question 1}

\section{Question 4: Repetition}
\begin{enumerate}
    \item Given we have the following predicates and functions with their corresponding meanings
    \begin{itemize}
        \item $D(x,y)$: $x$ is the defense attorney of case $y$;
        \item $P(x,y)$: $x$ is the prosecutor of case $y$;
        \item $J(x,y)$: $x$ judges case $y$;
        \item $W(x,y)$: $x$ is a witness in case $y$;
        \item $d(x)$: returns the defendant of case $x$.
        \item $w(x)$: returns the star-witness of case $x$;
        \item $c(x)$: returns the last case person $x$ was the star-witness in;
    \end{itemize}
    Give translations to first-order logic of the following sentences using only the predicates and functions given (and possibly the $=$-sign). You may introduce constants to denote proper names where necessary
    \begin{enumerate}
        \item Phoenix is a defense attorney
        \begin{itemize}
            \item $\exists x \exists y (D(x,y) ^ (x = Phoenix))$
        \end{itemize}
        \item Lotta is not a witness in case Skye
        \begin{itemize}
            \item $\exists x \exists y (\neg W(x,y) ^ (x = Lotta, y = Skye))$. The predicate can also be written as: $\exists x \exists y (\neg W(L, S))$, however this is a bit more unclear as the constants are not declared and $x$ and $y$ are not specifically used in the predicate W.
        \end{itemize}
        \item The Fey case and the Engarde case featured the same star witness.
        \begin{itemize}
            \item $\exists x \exists y \exists z ((W(x, y) \land W(x, y) \land (x = (w(y) \land w(x)) \land y = Fey, z = engarde \land (y \neq z))))$
        \end{itemize}
        \item Nobody is the defense attorney for a case that (s)he prosecuted.
        \begin{itemize}
            \item $\forall x \exists y ( D(x,y) \to \neg P(x,y))$
        \end{itemize}
        \item There is someone who is now the defendant of a case, after being the star-witness in another case.
        \begin{itemize}
            \item $\exists x \exists y \exists z (D(x,y) \land W(x,z ^ (y \neq z \land x = w(x)))) $
        \end{itemize}
        \item Godot has prosecuted exactly one case.
        \begin{itemize}
            \item $\exists x \exists y (P(x,y) \land (x = godot) \land \forall z (P(x,z) \to y = z))$
        \end{itemize}
        \item People who have been a defendant in at least one case always end as a defense attorney, prosecutor or judge in another case.
        \begin{itemize}
            \item 
        \end{itemize}
    \end{enumerate}
    \item Give the negation of each of the following formulae.
    \begin{enumerate}
        \item $\exists x\forall y(P(x,y)\iff Q(x,y))$.
        \begin{itemize}
            \item $\equiv \exists x\forall y((P(x,y) \to Q(x,y)) \land (Q(x,y) \to P(x,y)))$
            \item $\neg (\exists x\forall y((P(x,y) \to Q(x,y)) \land (Q(x,y) \to P(x,y))))$
            \item $\equiv \forall x\exists y((\neg P(x,y) \to \neg Q(x,y)) \lor (\neg Q(x,y) \to \neg P(x,y)))$
            \item $\equiv \forall x\exists y(( P(x,y) \land \neg Q(x,y)) \lor (Q(x,y) \land \neg P(x,y)))$
        \end{itemize}
        \item $\forall y\exists x(P(x,y)\to\neg\exists x(Q(x,y)\vee(x=y)))$.
        \begin{itemize}
            \item $\equiv \forall y\exists x(\neg P(x,y)\lor \neg\exists x(Q(x,y)\vee(x=y)))$
            \item $\neg (\forall y\exists x(\neg P(x,y)\lor \neg\exists x(Q(x,y)\vee(x=y))))$
            \item $\equiv \exists y\forall x( P(x,y)\land \neg\forall x(\neg Q(x,y)\land (x \neq y)))$
        \end{itemize}
        \item $\exists x(R(x)\vee\forall y(P(x,y)\wedge(f(x,y)=y)))$.
        \begin{itemize}
            \item $ \neg (\exists x(R(x)\vee\forall y(P(x,y)\wedge(f(x,y)=y))))$
            \item $\equiv \forall x(\neg R(x)\land\exists y(\neg P(x,y)\lor (f(x,y) \neq y)))$.
        \end{itemize}
        \item $\exists x\forall y(\neg P(x,y)\iff\neg Q(x,y))$. 
        \begin{itemize}
            \item $\equiv \exists x\forall y((\neg P(x,y) \to \neg Q(x,y) \land (\neg Q(x,y) \to P(x,y))))$
            \item $\equiv \exists x\forall y(( P(x,y) \lor \neg Q(x,y) \land ( Q(x,y) \lor P(x,y))))$
            \item $\neg (\exists x\forall y(( P(x,y) \lor \neg Q(x,y) \land ( Q(x,y) \lor P(x,y)))))$.
            \item $\equiv \forall x\exists y(( \neg P(x,y) \land  Q(x,y) \lor ( \neg Q(x,y) \land P(x,y))))$.
        \end{itemize}
        \item $\forall x\forall y\exists z(S(x,y,z)\to(R(z)\vee\neg P(z)))$.
        \begin{itemize}
            \item $\equiv \forall x\forall y\exists z(\neg S(x,y,z)\lor(R(z)\vee\neg P(z)))$
            \item $\neg (\forall x\forall y\exists z(\neg S(x,y,z)\lor(R(z)\vee\neg P(z))))$
            \item $\equiv (\exists x\exists y\forall z( S(x,y,z)\land(\neg R(z)\vee P(z))))$
        \end{itemize}
    \end{enumerate}
    \item Suppose we have a structure $\mathcal{A}$, with the domain $D=\{a,b,c,d,e\}$ and predicates $P=\{a,d\}$, $Q=\{(a,b),(d,e),(a,c),(b,c)\}$ and $R=\{(a,d),(b,e),(c,a),(d,b),(e,c)\}$.

    For each of the following formulas, indicate whether they are true in $\mathcal{A}$. If a formula is true, explain why, and if it is not, give a suitable counterexample
    \begin{enumerate}
        \item $\forall x(P(x)\to\exists y(Q(x,y)))$.\\
        The given formula is True as there does exist a tuple for every value of predicate $P$ of which that value is the first value of the tuple with any given $y$ value.
        \item $\forall x\forall y(Q(x,y)\to\neg R(x,y))$.\\
        The stated formula is true as there does not exist any combination of the values $x$ and $y$ in predicate $Q(x,y)$ which is also present in predicate $R(x,y)$, hence the statement is true (on the basis of the rules of an implication).
        \item $\forall x(P(x)\to\forall y(Q(x,y)\vee\neg R(x,y)))$. \\
        The given statement is True as for any value in predicate $P(x)$ there does exists a tuple with said $x$ at the first position of the tuple and any other given $y$, which are: $\{(a,b), (a,c), (d,e)\} $.
        There does also exist a tuple in predicate $R(x,y)$, however due to the negation and the $\lor$ operator these can be left out as the other statement already makes the formula true.
        \item $(\neg P(a)\vee\neg\exists x\forall yQ(x,y))\iff\exists x(Q(a,x)\wedge R(c,x))$. \\
        The given formula is False on the basis that one sides of the bi-directional implication is false with the other being true (rules of a bi-directional implication).
        To start, $\neg P(a)$ is false as there does exist a value $a$ in the list of predicate $P$.
        Second, the statement $\neg (\exists x\forall yQ((x,y)))$ is false (as there does not exists a value which is present as $x$ for all other values of the domain in the $y$ position). However due to the negation it becomes true, this makes the Left Hand Side of the statement true.
        However, the Right Hand Side of the statement is False as there does not exist a value for $x$ which is present in a tuple with value $a$ on the first position in predicate $R(a,x)$ and the same is true for predicate $R(c,x)$.
        As the RHS is False with the RHS being true for the given formula, hence the formula is False (based om the rules of a bi-directional implication).
        \item $(P(d)\wedge\exists yR(y,d))\to\forall x(R(x,c))$.\\
        The given formula is False.
        To start, the value $d$ is valid for predicate $P$, there does also exist a value for $y$ for which the statement $R(y,d)$ is true, the statement is true as predicate $R$ contains the tuple $(a,d)$.
        This makes the Left Hand Side (LHS) of the statement true.
        However, the Right Hand Side (RHS) is False as there do not exist tuples for all possible values of $x$ which have $c$ as the second value for predicate $R(x,c)$.
        As the RHS needs to be true as it is necessary for the formula to hold (rules of an implication), therefore the formula is False.
    \end{enumerate}
\end{enumerate}
\end{document}
