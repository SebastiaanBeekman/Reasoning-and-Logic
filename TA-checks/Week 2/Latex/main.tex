\documentclass[a4paper]{article}

\usepackage{graphicx} 
\usepackage[english]{babel}
\usepackage{graphicx}
\usepackage{float}
\usepackage{logicproof}
\usepackage{amssymb}
\usepackage[a4paper,top=3cm,bottom=2cm,left=2cm,right=2cm,marginparwidth=1.75cm]{geometry}

\begin{document}

\begin{titlepage}
    \newcommand{\HRule}{\rule{\linewidth}{0.5mm}}
    \center

    \textsc{\LARGE Delft University of Technology}\\[1cm]

    \textsc{\Large Reasoning \& Logic}\\[0.2cm]
    \textsc{\large CSE1300}\\[1cm]
    \HRule \\[0.8cm]
    { \huge \bfseries Assignment: TA-check 2}\\[0.7cm]
    \HRule \\[2cm]
    \large
    \emph{Authors:}\\
    Joris Rijs (5880998) \& Sebastiaan Beekman (5885116)\\[1.5cm]
    {\large \today}\\[5cm]
    \includegraphics[width=0.6\textwidth]{images/TU_delft_logo.jpg}\\[1cm]
    \vfill
\end{titlepage}

\newpage
\tableofcontents

\newpage
\section{Question 1}

For each of the following statements, indicate whether they are \textit{necessary} for $8 | n$ with \textit{n} an arbitrary integer. Give a convincing argument for your answer.
\begin{itemize}
    \item $ n > 3  $ \\
    Yes this attribute is necessary, however as the given statement works with integers from which we assume that the answer should also be an integer.
    With this in mind, any integer which is smaller than 8 (which also includes the limiter 3) will result in an outcome which is not an integer.
    Knowing this, n should be greater or equal to 8,this will hold as the result of $8 \mid 8 = 1$ which results in a prime factor.
    % No, as one of the prime factors for 8 which where n \s 3 is 2.
    % This is on the basis that 8 is divisible by 2 as well as 2 being a prime number, this makes numbers n smaller than
    \item $ 512 | n^3 $ \\
    Yes this statement will hold however it is not necessary for the original statement, as the given statement can be simplified by taking the cube root of 512.
    This equates to $8 \mid n $, which is equivelant to the original statement.
    \item $ 7 \nmid n $ \\
    No, this statement is not necessary, this does not add anything to the original statement.
    Another argument for the missing necessity is that there are arbitrary \textit{n} values which are both devisable by 8 and 7 at the same time.
    This is on the premise that any arbitrary integer is allowed to be chosen for \textit{n} .
\end{itemize} 

For each of the following statements, indicate whether they are \textit{sufficient} for $5 | n$.
As a general simplification, the base statement can be rewritten as \textit{n = 5 * k} which states that any number divisable by 5 is any number which is a multiple of 5.

\begin{itemize}
    \item $ 5  \sqrt[2]{n} $ \\
    The given statement is not valid. if we use the simplification given it is not possible to get a multiplication which results in the result of 
    The only way to get the square root of 5 is to multiply 5 to the power 1/2, this will mathematically result in the same answer as square root of 5.
    \item $ 3 \nmid n $ \\
    To start, we can simplify the formula to n  3 * k where k is any arbitrary integer.
    On this basis we can say that the statement is not sufficient as that there are integers which are a multiple of 5 / are devisable by 5 which are also a multiple of 3 / devisable by 3.
    This makes the statement false thus not sufficient.
    \item $ n = 25^k (k \ge 3) $ \\ 
    The given statement is sufficient on the basis that $(25^k)$ is equivelant to $ ((5^2)^k) $ which is equivelant to $ (5^{2k}) $.
    This means that the base integer is 5 and thus the product of the given statement is devisable by 5 thus making both statements true, making the given statement sufficient.
    
\end{itemize}

\newpage
\section{Question 2}

\newpage
\section{Question 3}
For each of the following abstractly formulated claims, outline a proof for the claim. Highlight what proof techniques you would use in your proof. For example for : $ \forall x \ (P(x) \to Q(x))$, we would answer: Take an arbitrary x such that \textit(P(x)) holds. Now we show that \textit{R(x)} holds. Since x was arbitrarily chosen, it will hold for all x.\\

\textbf{a} $ \exists x \ (P(X))$\\ 
The given statement can be proven by existence.
This can be done by finding a value of x which holds, for example: there is an even prime number.
This can be formally written as:
{
    \noindent
    \setlength\subproofhorizspace{2em}
    \begin{logicproof}{1}
        \exists x \ (P(X)) & premise \\
        \exists x \ (P(X)) & premise \\\hspace*{-30pt}
        \textbf{True} & \textbf{conclusion} 
    \end{logicproof}
}

\textbf{b} $ \forall x \ (P(x) \iff Q(x))$\\
The given statement can be proven by using both generalization and contrapositive.
To start, the given statement can be simplified to $ \forall x \,((P(x) \to Q(x)) ^ (Q(x) \to P(x)))$.
This allows for the ability to proof the first and second statement separately.
we need to choose an arbitrary x for which \textit{P(x)} holds.
{
    \noindent
    \setlength\subproofhorizspace{2em}
    \begin{logicproof}{1}
        \forall x \, (P(x) \to Q(x)) & premise \\
        \forall x \, P(x) & premise \\\hspace*{-30pt}
        \forall x \, Q(x) & \textbf{conclusion} 
    \end{logicproof}
}
{
    \noindent
    \setlength\subproofhorizspace{2em}
    \begin{logicproof}{1}
        \forall x \, (Q(x) \to P(x)) & premise \\
        \forall x \, P(x) & premise \\\hspace*{-30pt}
        \forall x \, \neg Q(x) & \textbf{conclusion} 
    \end{logicproof}
}

\textbf{c}  $\forall x \ (\neg P(x) \to \neg Q(x))$\\
The given statement can be proven doing a proof by generalization.
First, the given statement can be rewritten to $ \forall x \, P(x) \vee \neg Q(x))$.
If we use an arbitrary value for x for which \textit{P(x)} holds, then the statement will hold as only one of the statements needs to be true for the proof to be valid.
This can formally be written as
{
    \noindent
    \setlength\subproofhorizspace{2em}
    \begin{logicproof}{1}
        \forall \forall x \, P(x) \vee \neg Q(x)) & premise \\
        \forall x \, P(x) & premise \\\hspace*{-30pt}
        \forall x \, \neg Q(x) & \textbf{conclusion} 
    \end{logicproof}
}

\textbf{d} $ \forall x \ (P(X) \to \exists y \ (R(x,y)))$  \\
The given statement can be proven by generalization and existence.
The structure of the statement is such that x is allowed to be any arbitrary variable $(\forall x)$.
Based on this we can mainly use a proof by existence by finding a y value for which $\exists y \,R(x, y)$ holds.
And as x is allowed to be any arbitrary value, the statement will be true. 
(This should also still be valid if the statement is rewritten to $\forall x \, \neg P(x) \vee \exists y \, (R(x,y))$ as when the second predicate is true and the first predicate is false due to the 'or' operator).
The statement can formally be written as:
{
    \noindent
    \setlength\subproofhorizspace{2em}
    \begin{logicproof}{1}
        \forall \forall x \, \neg P(x) \vee \exists y \, (R(x,y)) & premise \\
        \forall y \, R(x,y) & premise \\\hspace*{-30pt}
        \textbf{True} & \textbf{conclusion} 
    \end{logicproof}
}

\newpage
\section{Question 4}

\newpage
\section{Question 5}

\end{document}